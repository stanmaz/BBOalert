\chapter{L'ouverture de 1\T}



\begin{multicols}{2}


L'ouverture de 1\T indique au moins 4 cartes à \C et au moins 12H. L'enchère est forcing et illimitée.

La réponse la plus négative est 1\P, toutes les autres réponses promettent 5H ou un As. Cette réponse de 1\P couvre deux cas, les mains très faibles et les mains forcing de manche sans autre enchère à disposition.

Il n'y a pas de notion de fit différé (à l'exception des mains très faibles). Si le répondant est fitté à \C, il doit le dire immédiatement. Avec un jeu banal pour jouer la manche, il répond 2\NT et l'ouvreur avisera. Avec une main limite de manche ou une main de chelem, il utilise le Texas à 2\K. Avec un jeu faible, mais constructif, il donne un soutien direct. Avec un jeu vraiment faible, le répondant commence par faire l'enchère très négative de 1\P.

Attention aux splinters. Les splinters en face d'une ouverture de 4 cartes sont plus forts que les splinters en face d'une ouverture de 5 cartes. On n'est pas en sécurité distributionnelle.

Sans fit, et jusqu'à 11H, le répondant répond 1\K sans 4 cartes à \P et 1\C sinon. A partir de 12H, il nomme sa couleur en Texas ou commence par le relai à 1\P s'il n'a pas de couleur intéressante. Attention, comme dans tout le reste du système, nommer une majeure quatrième est prioritaire même en présence d'une mineure plus longue.

Les changements de couleur à saut permettent de décrire en une seule enchère des unicolore limite de manche. Très pratique. On vend la main en une seule enchère.

L'ouverture de 1\T est à la fois la plus fréquente et la plus complexe.
\end{multicols}



\enchbox{Ouverture 1\T}
{
 1\K & 5-11H & Pas de majeure 4\ieme\\

 1\C & 5H+ & Au moins 4 cartes à \P\\

1\P & 0-4H & Toute distribution\\
  \rb & 5-7DH & 4+ cartes à \C \\
  & 12H+ & Ni majeure 4\ieme ni mineure 5\ieme\\
1\NT & 12H+ & 5 cartes à \T\\

 2\T & 12H+ & 5 cartes à \K\\

 2\K & 11DH+ & 4+ cartes à \C\\


 2\C& 8-10DH & 4+cartes à \C\\
  2\P & 9-11H &  6+ cartes à \P \\
 2\NT & 12-15H& 4(5) cartes à \C \\
 3\T  & 9-11H &  6+ cartes à \T \\

 3\K & 9-11H &  6+ cartes à \K \\
3\C &7-9H& 5 cartes à \C \\

   3\P &8-11H& Splinter chicane indéterminée\\
  3\NT &8-11H & Splinter singleton \P\\
  4\T &8-11H& Splinter singleton \T \\
 4\K &8-11H& Splinter singleton \K\\
 4\C && Barrage 6 cartes\\
  4\P & 0-9H & Naturel 7+ cartes\\
  4\NT && Blackwood brutal\\
}

%%%%%%%%%%%%%%%%%%%%%%%%%%%%%%%%%%%%%%%%%%%%%%%%%%%%%%%%%%%%%%%%%%%%%%%%%%%%%%%%%


\titre{
  1\T
  Réponses après intervention
}

\begin{multicols}{2}


En cas d'intervention, on conserve l'esprit du système. Toutefois, les Texas mineurs perdent leur caractère forcing de manche et peuvent se faire dès 8H avec une belle couleur.
Le seul schéma à bien mémoriser, c'est celui du contre d'appel. Comme l'enchère est fréquente, cela devrait se faire assez vite.


\section*{Intervention par \Double}

Le \Redouble est une enchère punitive. Il dénie 4 cartes à \C et promet 10H et deux couleurs 4\ieme. Tous les contre subséquents sont punitifs.
Alternativement, en fonction de la vulnérabilité, on peut préférer annoncer 3\NT ou faire un relai à 1\P pour faire jouer les \NT de la main de l'ouvreur. L'enchère de 1\P devenant exclusivement forcing de manche.

Avec 4 cartes à \C et une main de manche, on conserve la réponse à 2\NT.

Avec une main unicolore, à partir de 8H, on utilise une enchère Texas 1\NT ou 2\T. L'enchère Texas de 1\C promet 5 cartes à \P et non plus 4.

Avec une main fittée limite de manche ou ambition de chelem, on passe par le Texas à 2\K.
L'enchère de 2\C garde sa signification fittée 8-10 DH.

L'enchère de 1\K est artificielle. Elle promet au moins 7-8H et sert d'enchère poubelle. Pas envie de punir, pas envie de laisser jouer, pas de fit 4\ieme.
Une mineure cinquième de mauvaise qualité par exemple ou 4 cartes à pique. Le plus souvent, on va jouer à \NT mais on laisse la place au déclarant pour déclarer 5 cartes à \C le cas échéant. On laisse aussi un peu de place aux adversaires pour s'exprimer, ça sera utile pour le jeu de la carte.


\section*{Intervention par 1\K}
Lorsque l'adversaire intervient naturellement, on va apporter quelques modifications. D'une part le \Double remplace l'enchère poubelle de 1\K, à ceci près qu'elle promet 8H.
L'enchère Texas de 1\C n'est pas modifiée ni sa suite. L'enchère de 1\P devient exclusivement forcing de manche. Les enchères Texas de 1\NT et de 2\T ne promettent plus que 8H.
le répondant ne s'engage à reparler que s'il à 12H et plus. Un ouvreur dotté de 14H ou plus ne pourra pas se contenter d'une simple rectification. A partir de 2\K, le système est inchangé.


\section*{Intervention naturelle par 1\C}
le \Double est d'appel et recherche en priorité un fit à \P. L'enchère de 1\P devient exclusivement forcing de manche. Les enchères Texas de 1\NT et de 2\T ne promettent plus que 8H.
Les enchères de 2\K et de 2\C sont inchangées, on part du principe que les adversaires se sont trompé.

\section*{Intervention artificielle non bicolore par 1\C}
Que l'enchère soit naturelle ou artificielle, on ne change rien. il est impossible de prévoir ce que les adversaires peuvent inventer.
Si l'enchère de 1\C promet explicitement une longueur à \P, il est clair que le \Double est plutôt à la recherche d'une mineure\dots

\section*{Intervention par 1\P}
Le \Double promet 8H et n'est plus limité à 11H puisqu'on ne dispose plus de l'enchère de 1\P. Les enchères Texas de 1\NT et de 2\T ne promettent plus que 8H.
L'enchère de 2\K montre une main limite de manche ou à la recherche d'un chelem. Le cue-bid indique un jeu de manche sans, a priori, d'espoir de chelem.

\section*{Intervention par 1\NT}

Le contre est punitif, à partir de (9)10H. L'enchère nous prive de Texas. Toutes les enchères au niveau de 2 sont naturelles non forcing. Il est possible de rater un fit à \P.

La seule difficulté est la suite à donner après un contre. Les adversaires peuvent continuer de façon naturelle ou en Texas. Il convient d'être précis sur le caractère punitif du \Double.
La règle utilisée ici est la suivante : «\textit{Le contre de toute enchère nommée naturellement est punitif}»

Lorsque les adversaires sauvent le contrat, il faut être clair sur les priorités : Avec 5 cartes à \C, l'ouvreur doit en priorité rechercher un fit 5-3. Avec 4 cartes à \P, il faut soit punir avec 4 cartes dans leur couleur, soit rechercher le fit \P. Avec 3 cartes à \P et en acescence de fit \C, il arrivera parfois de devoir contrer punitivement en réveil avec seulement 3 atouts. La vulnérabilité a beaucoup d'importance quant à la conduite à tenir ici.



\enchbox{1\T - 1\NT - \Double - 2\T (Texas)}
{
 \Pass & Banal Ou punitif à \K\\
 \Double & Appel \\
 2\K & Pas envie de punir\\
 2\C & 5 cartes 15-17H\\
 2\P & Bicolore cher\\
 3\T & Canapé

}

Si l'enchère est naturelle

\enchbox{1\T - 1SA - \Double - 2\T (Naturel)}
{
 \Pass & Appel\\
 \Double & Punitif\\
 2\K & Canapé\\
 2\C & 5 cartes 15-17H\\
 2\P & Bicolore cher\\
 3\T & Pas envie de punir
}



\section*{Intervention 2\T ou 2\K}

A partir de 2\T, on oublie les Texas. Le \Double est Spoutnik et promet 4 cartes à \P.
L'enchère de 2\K devient naturelle, forcing un tour, de même que l'enchère de 2\P.
Et l'enchère de 3\C indique un soutien limite de manche.


\section*{Intervention bicolore}
Les enchères naturelles sont non forcing. Les cue-bids sont forcing de manche. le cue-bid économique est fitté et le deuxième cue-bid indique 5+ cartes dans la quatrième couleur.

\section*{Réveil de l'ouvreur}
Court dans la couleur adverse, le \Double vient bien.
Avec un Canapé, on le nomme.

\end{multicols}


%\section*{Attitude de l'ouvreur}

%%%%%%%%%%%%%%%%%%%%%%%%%%%%%%%%%%%%%%%%%%%%%%%%%%%%%%%%%%%%%%%%


\titre{
  1\T -- 1\K --}


Le répondant a montré un jeu de 5 à 11H.
Les enchères à saut (sauf 2\NT) sont forcing de manche.

La redemande de 1\NT indique une main de 15-17H. Avec une main régulière de 12 à 14H ou de 22H+, on redemande 1\P, enchère Texas pour 1\NT. Cette façon de faire permet, le plus souvent de faire jouer la main du même coté que le champ, aussi bien sur l'ouverture de 1\T que sur celle de 1\K.

Par ailleurs, la redemande à 1\P permet aussi d'annoncer les jeux forts non forcing de manche.

\textit{Avec un jeu régulier de 15-17H et 5 cartes à \C, il y a débat. Faut-il répondre 1\C ou 1\NT ? Dans l'état actuel du système, on répond 1\NT de sorte que 1\C montre un jeu excentré ou de 18H+}

\enchbox{1\T -- 1\K}
{
 1\C & 15H+ & 5+ cartes à \C\\
 1\P & 12H-14H & Régulier\\
 \rw & 15H+ & Excentré, exactement 4 cartes à \C \\
 & 18H+ & Exactement 4 cartes à \C\\
 1\NT & 15-17H & Régulier\\
 2\T & 12-14H & Canapé\\
 2\K & 12-14H & Canapé\\
 2\C & 20H+ & 5+ cartes à \C Forcing de manche\\
 2\P & 20H+ & 4 cartes à \C Forcing de manche\\
 2\NT & 18-19H & Régulier 4 cartes à \C\\
 3\T &20H+& Naturel Forcing de manche\\
 3\K &20H+& Naturel Forcing de manche\\
 3\C &20H+ & 6+ cartes à \C Forcing de manche\\
 3\P & 18H+ & 6-5 forcing de manche \\
}

%%%%%%%%%%%%%%%%%%%%%%%%%%%%%%%%%%%%%%%%%%%%%%%%%%%%%%%%%%%%%%%%

\titre{1\T--1\K--1\C}

\begin{multicols}{2}


La redemande de 1\C indique au moins 5 cartes et va de 15 à 19H. La réponse de 1\K a fortement limité les possibilités de chelem. Et la manche reste incertaine. L'enchère est forcing un tour.

Avec un jeu faible, le répondant fait la redemande habituelle à 1\P. Il faut la même enchère à partir de 9H sans enchère naturelle.

Avec 3 cartes à \C, donner le fit est obligatoire (sauf 5-6H). Le répondant redemande alors 2\C, 3\C ou 4\C en fonction de sa force. Au delà de 2\C, la situation étant forcing de manche, c'est l'enchère de 3\C la plus forte.

\enchbox{1\T--1\K--1\C}
{
1\P & 5-6 ou 9+ & Faible ou banal\\
1\NT & 7-8H & Régulier \\
2\T  & 7-8H & 6 cartes \\
2\K  & 7-8H & 6 cartes \\
2\C  & 7-8DH & 3 cartes \\
3\C  & 11DH+ & 3 cartes \\
4\C  & 9-10DH & 3 cartes\\
}

Sur 1\P, l'ouvreur peut nommer les \NT, il peut faire un Canapé. Il peut même annoncer 2\P avec un 6-5 faible.

Sur 1SA, l'ouvreur doit faire un saut s'il veut imposer la manche.

\end{multicols}

\titre{1\T--1\K -- 1\P}

\begin{multicols}{2}

L'enchère de 1\P est un  Puppet pour 1\NT. Mais le répondant peut refuser de rectifier. Il y a deux situations où il va refuser.
Avec 11H, il rectifie à 2\NT au lieu de 1\NT pour ne pas rater la manche en face de 14H.

Avec une mineure longue, il la nomme. (La main ne peut pas être forte, sinon, il l'aurait nommé au tour précédent)

Le plus souvent, le répondant se contente de rectifier à 1\NT. Maintenant, en reparlant, l'ouvreur peut montrer un jeu fort. Tout est naturel.

\enchbox{1\T--1\K -- 1\P}
{
1\NT & 5-10H & Régulier\\
2\T & 5-10H & (5)6 cartes \\
2\K & 5-10H & (5)6 cartes \\
2\C & 5-9H & Bicolore mineur \\
2\P & 10-11H & Bicolore mineur \\
2\NT & 11H & Régulier\\
3\T & 5-8H & 7 cartes \\
3\K & 5-8H & 7 cartes \\
}

\enchbox{1\T--1\K -- 1\P--1\NT}{
2\T & 15-17H & Canapé\\
2\K & 15-17H & Canapé \\
2\C & 18+H & 4-4-4-1\\
\rb    & relai & 2\P quasi-obligatoire\\
\rb -> & 2\NT& Singleton \P\\
\rb -> & 3\T & Singleton \T\\
\rb -> & 3\K & Singleton \K \\
2\P &  & Bicolore 6-5 \\
2\NT & 22H+ & Régulier \\
3\T & 18-19H & Canapé \\
3\K & 18-19H & Canapé \\
}

Sans enchère naturelle, l'ouvreur peut passer par 2\C. La main est 4-4-4-1 mais assez mal zonée. Le singleton peut être à pique ou en mineure. Le répondant interroge le singleton par un relai à 2\P.



\end{multicols}


\titre{1\T--1\K--1\NT}

On se retrouve avec l'équivalent d'une ouverture de 1\NT fort. A ceci près que le problème des fits majeurs est déjà résolu. Il n'y en a pas.

De plus, avec un répondant limité à 11H, on ne voit pas bien ce qu'on irait faire en mineure !

Le répondant peut passer, proposer la manche en disant 2\NT (ou 2\C avec 3 cartes) ou conclure à 3\NT.

Avec une belle mineure longue, le répondant l'aurait nommée au tour précédent. Les enchères de 2\T et 2\K sont donc logiquement non forcing. Avec les deux mineures et 9-11H, on peut utiliser l'enchère de 2\P. 

Avec un singleton dans une main de 10--11H, le répondant peut avoir envie d'éviter un mauvais 3\NT. Mise à part l'enchère de 3\C qui est une enchère de politesse, toute enchère au niveau de 3 montrent un singleton.


\titre{1\T--1\K--2\NT}

Sur toute redemande de l'ouvreur à 2\NT, 3\T est un check-back Stayman (inutile dans cette séquence particulière mais on ne veut pas créer d'exception).

Les autres enchères au niveau de 3 montrent un singleton (3\C dans la mejeur de l'ouvreur pour le singleton \T)


\titre{
  1\T -- 1\C --}

Le répondant a montré 4 cartes à  \P.

L'enchère de 1\C dévore un peu d'espace. Cela a deux conséquences. Le relai à 1\P devient moins précis. Il montrera le plus souvent une main régulière de 12-14H non fitté à \P. mais il peut cacher un fit \P forcing de manche ainsi que tout un tas de mains fortes. D'autre part, la réponse de 1SA peut masquer 5 cartes à \C.

\enchbox{1\T -- 1\C --}
{

 1\P & 12H+ & Relai fourre-tout\\
 1\NT & 15-17H  & Régulier\\
 2\T & 12-14H & Canapé\\
 2\K & 12-14H & Canapé\\
 2\C & 15-17H & 5 cartes à \C, irrégulier \\
 2\P & 12-14H & 4 cartes à \P\\
 2\NT & 18-19H& Régulier\\
 3\T &(18)19H+& Naturel Forcing de manche\\
 3\K &(18)19H+& Naturel Forcing de manche\\
 3\C & 19H+ & 6 cartes à \C Forcing de manche\\
 3\P & 17-19DH & 4 cartes à \P\\
}

\titre{1\T--1\C--1\P}

\begin{multicols}{2}



C'est la séquence la plus ambiguë.

Le fit \C est exclu mais le fit \P est encore possible. Si l'ouvreur a une main forte, il le fera savoir. Une redemande ultérieure à \P de l'ouvreur montrera une main très forte (au moins 20 points) et sera forcing de manche.

Avec 5 cartes à \P, le répondant répète ses piques avec un jeu faible ou passe par l'enchère de 2\C avec au moins 11H.

Les enchères en mineure sont des Canapés. Et tout ce qui est au niveau de 3 est forcing de manche.

\enchbox{1\T--1\C--1\P}
{
1\NT & 5-10H & 4 cartes à \P \\
2\T  & 5-10H & 5+ cartes à \T \\
2\K  & 5-10H & 5+ cartes à \K \\
2\C  & 11H + & 5+ cartes à \P \\
2\P  & 5-10H & 5+ cartes à \P \\
2\NT & 11H & Fourre-tout \\
3\T & 12H+ & 5+ cartes à \T \\
3\K & 12H+ & 5+ cartes à \K \\
3\C & 16H+ & Régulier \\
3\P & 15H+ & 6 belles cartes à \P \\
3SA & 13-15H & Régulier \\
}


\end{multicols}

\titre{1\T--1\C--1\P--1\NT}

\begin{multicols}{2}
 Le plus souvent, avec une main banale de 12-14H, l'ouvreur va passer. Toutefois, il peut avoir tout un panel de mains fortes, voire très fortes.
 Le principe est le suivant : toute enchère à saut est forcing de manche, soit dans une optique de chelem, soit dans une optique de recherche de la meilleure manche.
 Toute enchère sans saut indique la zone au dessus que la même enchère non précédée de 1\P.

 \enchbox{1T--1\C--1\P--1\NT}
 {
 2\T & 15-17H(18) & 5 cartes à \T \\
 2\K & 15-17H(18) & 5 cartes à \K \\
 2\C & 18-19H & 5 cartes à \C \\
 2\P & 18-19H & 4 cartes à \P \\
 2\NT & 22-23H & Régulier \\
 }

\end{multicols}

\titre{1\T--1\C--1\NT}

\begin{multicols}{2}
L'ouvreur possède 15-17H régulier avec 4 ou 5 cartes à \C. Et le répondant possède au moins 4 cartes à \P. Un fit reste donc possible dans chaque majeure.

De plus, la main du répondant est illimitée.

Une chose est sûre, l'enchère de 2\C est une réitération du Texas avec au moins 5 cartes à pique et l'enchère de 2\NT est une proposition de manche

La nomination d'une mineure est canapé et dénie 5 cartes à pique.

Si le répondant hésite à jouer 3SA, il montre son singleton au niveau de 3.

\enchbox{1\T--1\C--1\NT}
{
2\T &5-7H& 5 cartes faible\\
2\K &5-7H& 5 cartes faible\\
2\C& 8H+& 5 cartes à \P\\
2\P& 5-7H & 5 cartes à \P\\
2\NT & 8H & Naturel \\
3\T & 9H+ & Singleton \T\\
3\K & 9H+ & Singleton \K\\
3\C & 9H+ & 3 cartes à \C\\
3\P & 16H+ & 6 cartes à \P \\
3\NT && Conclusion\\
}



\end{multicols}


\titre{1\T--1\C--2\NT}

Sur toute redemande de l'ouvreur à 2\NT, 3\T est un check-back Stayman à la recherche d'un soutien de 3 cartes à \P. L'enchère de 3\P montre 6 cartes et des ambitions.

Les autres enchères au niveau de 3 montrent un singleton (3\C dans la mejeur de l'ouvreur pour le singleton \T)





\titre{
  1\T -- 1\P --}

La réponse de 1\P est un signal d'alarme. La main peut être très faible. Et même si elle est forte, il ne faudra pas en attendre plus qu'un tas de points H.
L'ouvreur considère dans un premier temps que l'enchère est faible et avisera le cas échéant.

L'enchère de 2\P montre l'équivalent d'un deux fort indéterminé.

Les sauts sont forcing de manche à la recherche du meilleur contrat, jouable en face d'une main nulle.




\enchbox{1\T -- 1\P}
{
 1\NT & 12H-19H & Régulier\\
 2\T & 12-21H & Canapé\\
 2\K & 12-21H & Canapé\\
 2\C & 15-21H & 5 cartes à \C, irrégulier \\
 2\P & 22-23H & Tout type de main, relai obligatoire à 2\NT\\
 \rb-> & 3\T & Naturel non forcing \\
 -> & 3\K & Naturel non forcing \\
 \rb-> & 3\C & Naturel non forcing \\
 2\NT & 22-23H & Régulier\\
 3\T & & Naturel Forcing de manche\\
 3\K & & Naturel Forcing de manche\\
 3\C & & 6 cartes à \C Forcing de manche\\
 3\P & & Bicolore 6-5 Forcing de manche\\
}

%%%%%%%%%%%%%%%%%%%%%%%%%%%%%%%%%%%%%%%%%%%%%%%%%%%%%%%%%
\titre{1\T--1\P--1\NT}

\enchbox{1\T--1\P--1\NT}
{
2\T & 0-4H & (5)6 cartes \\
2\K & 0-4H & (5)6 cartes \\
2\C & 0-7DH & 4 cartes à \C \\
2\P & 0-4H & 5+ cartes à \P \\
2\NT & 14H+ & Quantitatif\\
-> & 3\T & 12-13H\\
-> & 3\K & 14-15H\\
-> & 3\C & 16-17H\\
-> & 3\P & 18-19H\\
}

%%%%%%%%%%%%%%%%%%%%%%%%%%%%%%%%%%%%%%%%%%%%%%%%%%%%%%%%%
\titre{1\T--1\P--2\T}

Avec un jeu faible, le répondant passe ou rectifie à 2\C. Toute autre enchère est forcing de manche.

Le répondant a deux problèmes à résoudre. Connaître la force de l'ouvreur et connaître son singleton.

Le cas le plus courant, 12-13H avec les arrêts \K et \P, permet de conclure à 3\NT.





%%%%%%%%%%%%%%%%%%%%%%%%%%%%%%%%%%%%%%%%%%%%%%%%%%%%%%%%%%%%%%%%

\titre{1\T--1\P--2\C}

Cette séquence comporte un petit piège.
Avec 4 cartes à \C un jeu de 6-8DH, le répondant doit faire un petit effort à 3\C où, de toute façon il est en sécurité distributionnelle.
L'ouvreur ne doit pas croire à une main de chelem. La réponse de 1\P ayant exclu cette ambition. Si chelem, il y a, l'initiative viendra de l'ouvreur.

Avec un jeu fort (12H) et plat, le répondant peut utiliser l'enchère impossible de 2\P ou l'enchère naturelle de 2\NT. Il n'a que l'embarras du choix.

\titre{1\T--1\P--2\NT}
Sur toute redemande de l'ouvreur à 2\NT, 3\T est un check-back Stayman. On utilise les réponses standards. 

\enchbox{1\T--1\P--2\NT--3\T}
{
3\K && 5 cartes à \C et 3 cartes à \P\\
3\C && 5 cartes à \C et 3 cartes à \P\\
3\P && 4 cartes à \C et 2 cartes à \P\\
3\NT&& 4 cartes à \C et 2 cartes à \P\\
}

Les autres enchères au niveau de 3 montrent un singleton (3\C dans la mejeur de l'ouvreur pour le singleton \T)


\titre{
  1\T -- 1SA --}


L'enchère de 1SA est forcing de manche avec une vraie couleur \T. Par ailleurs, l'enchère dénie 4 cartes à \C et 4 cartes à \P.

Sans rien de particulier à dire, l'ouvreur se contente de rectifier à 2\T pour voir.
Avec 5 cartes à \C, il les nomme en disant 2\C. Les enchères de 2\K et 2\P sont naturelles et promettent 5 cartes.

\enchbox{1\T--1\NT}
{
2\T &12-14H & Fitté ou pas\\
2\K &12-14H & 5 cartes \\
2\C &15H+& 5 cartes \\
2\P &12H+& Bicolore 6-5 \\
2\NT &15H+& 4 cartes à \C\\
3\T  &15H+& Beau fit \\
3\K & 15H+& naturel \\
}

Sur le redemande à 3\T, le répondant nomme son singleton, y compris à \C. Sans singleton, il propose 3\NT. Avec un jeu exceptionnel sans singleton, il a le choix entre l'enchère de 4\T qui démarre directement les négociations de chelem et l'enchère quantitative de 4\NT.

Sur la redemande à 3\K, les enchères de 3\C et de 3\P sont aussi des singletons, le fit \K étant implicite. Non fitté, le répondant propose 3\NT (ou 4\NT quantitatif) ou insiste dans sa couleur.

%%%%%%%%%%%%%%%%%%%%%%%%%%%%%%%%%%%%%%%%%%%%%%%%%%%%%%%%%%%%%%%%

\titre{1\T--2\NT}

En annonçant 2\NT, le répondant annonce une main banale de manche sans ambition de chelems. le plus souvent, l'ouvreur conclu à 4\C.

Cependant, il n'a pas limité sa main. Toute autre enchère que 4\C est une tentative de chelem. L'enchère la plus forte est 3\C qui déclenche immédiatement la procédure de contrôles. Les enchères de 3\T et 3\K sont naturelles et demande au partenaire de juger sa main. Les enchères de
3\P, 4\T et de 4\K sont des splinters, pour que le partenaire juge sa main. \textit{Avec un bicolore majeur 6-5, rien de spécial n'est prévu, hors peut-être, le Blackwood d'exclusion à 5\T et 5\K pour les experts.}
