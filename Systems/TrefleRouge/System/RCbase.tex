\chapter{Les ouvertures}

\begin{multicols}{2}


\section*{Les  unicolores}


Attention, les mains 5-3-3-2 sont toujours considérées comme des mains régulières, même avec une majeure cinquième.
Les unicolores mineurs 6-3-2-2 de 12 à 14H s'ouvrent de 1SA.

Avec les piques, on ouvre de 1\P de 11H à 17H et de 1\K à partir de 18H. L'ouverture de 1\P est la seule ouverture agressive du système.
Avec un bicolore 5-5 à honneurs concentrés, on peut ouvrir avec 10 points d'honneur seulement.

Avec 6 cartes à \P, on ouvre de 2\P jusqu'à 10H et de 1\P à partir de  11H. Il n'y a pas de zone grise où on est trop fort pour ouvrir de 2\P et trop faible pour ouvrir de 1\P. En fonction de la position et de la vulnérabilité, la limite basse de l'ouverture de 2\P est élastique.

Avec les cœurs, on ouvre de 2\C de 12H à 14H et de 1\T au delà.

Avec les trèfles, on ouvre de 2\T de 12H à 14H et de 1\C au delà.


Avec les carreaux, on ouvre de 2\K de 12H à 14H et de 1\C au delà.

\section*{L'ouverture de 2SA}

Les mains régulières de 20 ou 21H s'ouvrent de 2SA. Cette ouverture est prioritaire même avec une majeure 4\ieme ou 5\ieme.

\section*{Les bicolores majeurs}

Dans la première zone de l'ouverture, de 12 à 14H, on annonce toujours les piques avant les cœurs. Donc avec 4 cartes à \P et 5 cartes à \C, on ouvre de 1\K.

Avec un jeu fort, 15H et plus, on annonce la couleur la plus longue en premier. On ouvre donc de 1\T avec les \C et de 1\P (ou de 1\K avec les jeux très forts, 18H+, avec les piques.)
En cas d'égalité, 5-5, ou 6-6, on nomme les \P en premier.

Avec un beau bicolore majeur, sans point perdu, on peut ouvrir à 11H et parfois même à 10H avec un beau bicolore 5-5.


\section*{Les bicolores mineurs}

Les mains 5-4-2-2 sont systématiquement assimilées à des mains régulières et n'entrent pas dans la cadre de ce paragraphe.
A partir de 15H, les bicolores mineurs s'ouvrent de 1\C.
De 12 à 14H, dans une main 5-4-3-1 ou 6-5, on ouvre dans la couleur la plus longue au niveau de 2.
Avec un bicolore 5-5 ou 6-6, on ouvre de 2\K.

\section*{Mixte majeur-mineur}

Qu'elle soit quatrième ou cinquième, \textbf{on nomme toujours la majeure en premier.}
Avec 5 cartes à cœur, une mineure quatrième et deux doubletons incluant des honneurs, dans une main de 12H à 14H, on pourra choisir d'ouvrir la main de 1SA.

\section*{Les mains régulières}

Les mains 4-4-4-1 suivent la même logique que les mains régulières.

Dans la zone 12-14H, on ouvre de 1\NT sans majeure 4\ieme (mais possiblement avec une majeure 5\ieme), on ouvre de 1\T suivi d'une redemande à 1\P avec 4 cartes à \C (et éventuellement 4 cartes à \P) et on ouvre de 1\K suivi d'une redemande à 1\P avec 4 cartes à \P sans 4 cartes à \C.

Dans la zone 15-17H, on ouvre de 1\C sans majeure 4\ieme, on ouvre de 1\T suivi d'une redemande à 1\NT avec 4 ou 5 cartes à \C et on ouvre de 1\K suivi d'une redemande à 1\NT avec 4 ou 5 cartes à \P.

Dans la zone 18-19H, on ouvre de 1\T, 1\K ou 1\C selon les majeures. La redemande est variable. On redemande 1\NT si le répondant a utilisé le signal de faiblesse à 1\P. Sur le relai ordinaire (1\K sur 1\T ou 1\C sur 1\K), on nomme la majeure au niveau de un avant de faire la redemande à 2\NT.

Dans la zone 20-21H, on ouvre de 2\NT.

Au dela de 22H, on ouvre au niveau de un et on fera une redemande à saut (forcing de manche) sauf sur le signal de faiblesse à 1\P qui décale la zone forcing de manche à 24HL.
\end{multicols}


 \enchbox{Tableau des ouvertures}{
 1\T & 12H+ & 4+ cartes à \C\\
 1\K & 12H+ & 4+ cartes à \P\\
 1\C & 15H+ & Pas de majeure 4\ieme\\
 1\P & 11-17H & 5 cartes à \P\\
 1\NT & 12-14H & Régulier\\
 2\T & 12-14H & 5+ cartes à \T\\
 2\K & 12-14H & 5+ cartes à \K\\
 2\C & 12-14H & 5+ cartes à \C, pas 4 cartes à \P\\
 2\P & 0-10H & 6 cartes à \P\\
 2\NT & 20-21H & Régulier\\
 3\T à 4\P &5-10H& Barrage constructif (sauf vert contre rouge)\\}




