\documentclass[a4paper,12pt, french]{book}
\usepackage[french]{babel}

\PassOptionsToPackage{table}{xcolor}
\usepackage{onedown}
\setdefaults{colors=4A, bidlong=off, bidline =0}
\usepackage{tcolorbox}
\newcommand{\T}{\Cl}
\newcommand{\K}{\Di}
\newcommand{\C}{\He}
\renewcommand{\P}{\Sp}
\usepackage{graphicx}
\usepackage{multicol}
\usepackage[Bjornstrup]{fncychap}

\usepackage{fontspec}
%\setmainfont{Universalis ADF Std}
%\setmainfont{liberation Sans}
\setmainfont{Caladea}
\newfontfamily{\myfont}{Caladea}[Ligatures = TeX]%[NFSSFamily=cmss]
\newfontfamily{\sansserif}{liberation Sans}[Ligatures = TeX]
\bidderfont{\mdseries\myfont}
\compassfont{\mdseries\sansserif}
\gamefont{\bfseries\sansserif}
\legendfont{\mdseries\sansserif}
\namefont{\mdseries\sansserif}
\otherfont{\bfseries\sansserif}


\definecolor{ashgrey}{rgb}{0.7, 0.75, 0.71}
\definecolor{apricot}{rgb}{0.98, 0.81, 0.69}
\definecolor{bananamania}{rgb}{0.98, 0.91, 0.71}

%\usepackage{fourier-otf}
\newcommand{\titre}[1]
{
\newpage
\begin{tcolorbox}[colback=red!5!white,colframe=red!75!black]

\begin{center}
\Large
  \includegraphics[width=0.5cm]{trefle.png}
  \hfill
  #1
  \hfill
  \includegraphics[width=0.5cm]{trefle.png}
\end{center}
\end{tcolorbox}
}

\newcommand{\enchbox}[2]{
\vspace{8pt}
\tcbox[left=0mm,right=0mm,top=0mm,bottom=0mm,boxsep=0mm,
toptitle=0.5mm,bottomtitle=0.5mm,title=#1]{%fond vert
\arrayrulecolor{blue!5!black}\renewcommand{\arraystretch}{1.2}%
\begin{tabular}{ccc}
 #2
\end{tabular}
}
\vspace{8pt}
}

\rowcolors{1}{bananamania}{white}
\newcommand{\rb}{\rowcolor{bananamania} }
\newcommand{\rw}{\rowcolor{white} }

\tcbset{colframe=red!50!black,colback=white,colupper=red!50!black,
fonttitle=\bfseries,nobeforeafter,center title}
\usepackage[firstpage]{draftwatermark}


\makeatletter
\renewcommand{\thesection}{\@arabic\c@section}
\makeatother

\begin{document}



 \begin{titlepage}
 \centering
 \SetWatermarkLightness{0.8}
\SetWatermarkText{\includegraphics[angle=-45]{REDCLUB.png}}
 {\huge\bfseries Le trèfle rouge\par}

 {\Large\itshape Hubert Quatreville\par}
 \vfill
\end{titlepage}


\section*{TODOLIST}

Ce document est une version de travail.
Un certain nombre de séquences risquent d'être modifiées.

[ ] Relecture par au moins deux bridgeurs

[X] Test sur BBO d'au moins 100 donnes

[ ] Test d'au moins 1000 donnes

[ ] Simplification des complexités inutiles

[ ] Confection d'une couverture attractive



\chapter*{Introduction}

\section{Pourquoi un nouveau système ?}

La théorie des enchères avance très vite.

Depuis la parution du best-seller de Pierre Jaïs et Michel Lebel en 1976 intitulé «La majeure cinquième» puis en 1982 «La nouvelle majeure cinquième» sous-titrée pompeusement «Le Bridge standard français», la majeure cinquième occupe une place de plus en plus hégémonique en France. On pourrait croire que ces ouvrages, de grande qualité, en sont la cause. Mais il n'en est rien puisque dans le monde entier, la majeure cinquième est devenue standard.

Et il y a une raison à cela. Avant la majeure cinquième, on jouait le plus souvent un système de type ACOL, nommé «La longue d'abord». Ce système est tombé aux oubliettes car à la fois plus difficile à manipuler et moins performant que la majeure par cinq. Toutefois, on lui doit beaucoup. En effet, conscient du caractère perfectible de ce système, beaucoup de champions ont développés des idées nouvelles. Tout particulièrement, un système nommé «Le Canapé» où, dans certaines situations, on annonce les bicolores en commençant par la couleur la plus courte. Le concept est resté ! Et plus important, le trèfle fort, un système qui consiste à ouvrir les mains fortes de 1\T ce qui permet de mieux zoner les autres ouvertures et, cerise sur le gâteau, de se débarrasser de l'ouverture de 2\T forcing de manche. Eric Rodwell, sept fois champion du monde, c'est un record, joue un système de trèfle fort.

De toute cette effervescence, c'est la majeure par cinq qui a émergé. j'ai relu l'ouvrage de Jaïs et Lebel de 1982. Tout est clair. Pas de gadgets. Juste quatre conventions en enchères à deux, le Blackwood, le Stayman, le Texas et la quatrième couleur forcing ; Et deux conventions, le contre d'appel et le contre Spoutnik en enchères à quatre. Et un principe simple : Quand on est fort, on fait un saut ou, à défaut, on annonce une nouvelle couleur.

Regardons comment a évolué le système français en 42 ans pour donner le SEF 2024. De nombreuses conventions se sont greffées au système.

Commençons par le Roudi. Le Roudi est un gadget, c'est à dire une enchère interrogative avec un système de réponses codifiées permettant à l'instigateur de prendre la bonne décision. Il entre en troisième position après le Blackwood et le Stayman. Ces gadgets sont très populaires. On apprend le système de réponse set on se débrouille. De même que le Stayman ou le Blackwood, le Roudi est un gadget indispensable dans tous les systèmes de majeure par cinq. Sans lui, on a un problème pour zoner les mains. Comme le Stayman, comme le Blackwood, le systèmes de réponses a évolué au fil du temps.

Le Roudi est important dans la théorie des enchères car il met bien en lumière comment un système se construit. On impose les premières enchères. On regarde ce qui se passe. Et quand on voit un trou dans la raquette, on ajoute un gadget. Au final, le joueur de club joue avec 4 ou 5 gadget couvrant les situations les plus fréquentes et le joueur de compétition, plus exigeant, va jouer avec une dizaine de gadget de façon à couvrir des situations moins fréquentes. Mais comme il joue souvent, cela revient grosso modo au même.

Il y a des alternatives au Roudi. Citons le double deux et le ping-pong. Ces deux conventions, à peu près équivalentes au Roudi en terme d'efficacité, sont terriblement impopulaires. Pourquoi ? Ce ne sont pas des gadgets ! D'ailleurs, en français, il n'y a pas de nom pour qualifier ces techniques. En anglais, on appelle cela des \textit{puppets} car le partenaire agit comme une marionnette. Le répondant commence par dire 2\T, l'ouvreur rectifie à 2\K (comme une marionnette) et le répondant s'explique. En français, le mot \textit{puppet}
est improprement utilisé pour décrire une façon particulière de jouer le Stayman. Bref, cette technique de \textit{puppet} est marginalisée. En effet, on transfert la décision au partenaire. Or comme chacun sait, le partenaire n'est pas digne de confiance \dots\ en général ! Bref, les gadgets sont beaucoup moins toxiques pour l'ego des joueurs que les puppets et ce n'est pas négligeable.

La deuxième convention que je souhaite examiner ici est le 2\NT fitté. C'est à mon sens, la plus grande révolution des enchères en 42 ans. Ce n'est pas un gadget ! C'est une enchère de moussaillon. Une enchère de moussaillon est une enchère où on vend sa main en une seule fois. On laisse alors son partenaire se débrouiller et on constate avec délectation que, contrairement à tous les pronostics, le brave bougre prend systématiquement les bonnes décisions. Le SEF 2024 regorge d'enchères de moussaillon, citons les enchères de rencontre ou les splinters.

Ce n'est pas la convention qui est importante mais ce sont ses conséquences. Avant le 2\NT fitté, le plus gros cauchemar des bridgeurs était de s'entendre sur la signification \textit{forcing } ou \textit{non forcing} de telle ou telle séquence. En 1984, Michel Perron, champion du monde, a publié un ouvrage de 57 pages intitulé «Que jouons-nous partenaire ?». Dans cet ouvrage il y a, entre autre, 32 séquences dont il faut préciser le caractère \textit{forcing} ou non en accord avec chaque partenaire. Une bonne école, mais fastidieuse.
Dans le SEF 2024, il suffit d'une demi-page pour expliquer ce qui est \textit{forcing}. En gros, tout ce qui ressemble à une enchère \textit{forcing} est une enchère \textit{forcing} ! Avec une unique exception, la séquence 1\C--1\P--2\C--3\C. Quel confort ! Et le deux sur un forcing de manche va dans le même sens. Moins d'exceptions, plus de confort. Quant à la séquence 1\C--1\P--2\C--3\C, on  ne peut pas la jouer \textit{forcing} \dots\ sans ajouter un gadget, par exemple en utilisant 2\P de façon conventionnelle; il n'y a donc pas de gain de simplicité à le faire.

Toujours est-il que les enchères évoluent de deux façons différentes et opposées. Ves une plus grande efficacité grâce à l'adoption de gadgets.
Chaque gadget permettant de combler  ou d'atténuer un inconvénient du système. Mais ce gain est coûteux. Il faut apprendre le système de réponses de ces gadgets. Et comme ces systèmes évoluent dans le temps, il faut vérifier qu'on a le même que le partenaire. Il y a même des systèmes de réponses qui servent de marqueurs sociaux. Typiquement le Blackwood 41-30 au lieu de 30-41. En région lyonnaise par exemple, jouer le Blackwood 30-41 vous disqualifie aux yeux des joueurs de compétitions car «comme chacun sait, le 41-30 est meilleur !».
Et je ne parle pas de la quatrième et de la troisième couleur forcing qui, une fois sur dix, mènent à des situations vraiment difficiles !
Le joueur malin, n'en a cure, quand la situation lui échappe, il propose 3\NT, au pire, il chutera comme les autres.

Les meilleures innovations sont celles qui amènent de la simplicité. Mais ce sont les plus difficiles. Il y a beaucoup plus d'intelligence derrière le 2\NT fitté que derrière le Roudi. Non seulement, les innovations simplificatrices sont difficiles à concevoir, mais pire, il est difficile de convaincre les bridgeurs de leurs intérêts. En effet, les bridgeurs ayant mis des années à maîtriser une méthodes complexes ne voient pas l'intérêt de passer à quelque chose de plus simple mais qui les oblige à changer leurs habitudes. Ils préfèrent, et de loin, le dernier gadget à la mode, préconisé par leur gourou local. Et c'est ainsi que le deux sur un forcing de manche a du mal à s'imposer. Et l'ouvrage d'Alain Levy, «Le Système Y» ne va certainement pas améliorer la situation. Ce qu'Alain Levy propose est beaucoup trop avancé. Il propose une méthode de champion du monde \dots\ pour champion du monde. Terriblement efficace et terriblement compliquée. Tous les gains en simplicité du nouveau système sont absorbés par l'ajout de petits gadgets, certes utiles et intelligents, voire fortement optimisés, mais qui rendent difficile l'apprentissage du système dans son ensemble.
Pour que les bridgeurs adoptent le deux sur un, il faut leur proposer un système basique, plus simple que le SEF, déjà très avancé, et aussi efficace que celui-ci. Ensuite, dans un deuxième temps, et pour ceux qui veulent, on ajoute les gadgets petit à petit.

En tout cas, aussi bien le 2\NT fitté que le deux sur un forcing de manche font apparaître une vérité. Une loi ! Un peu comme la loi des levées totales. Pour simplifier un système d'enchère, il faut le modifier en amont, pas en aval. En aval, on ajoute des gadgets. Et chaque gain en efficacité se fait au prix de la complexité. Par contre, une modification en amont peut amener soit de la simplicité à efficacité égale, soit de l'efficacité à complexité égale.

Oui mais comment faire les changements nécessaires ?

L'idée fondatrice qui a présidé à l'élaboration du Trèfle Rouge consiste à aligner le système d'enchères sur la stratégie des enchères. Je vais m'expliquer sur ce point car c'est la base de tout.

On trouve sur la toile un site amusant et très conne en France «Bridge Academy». Toute les semaines, le site examine une séquence d'enchère et vend un petit polycopié sur le sujet. A chaque fois, il rappelle la stratégie des enchères. Et il applique cette stratégie à la séquence
étudiée pour en déduire la conduite à tenir. Dans la plupart des cas, on va trouver un ou deux gadgets permettant de gérer les situations les plus épineuses. Avec une séquence par semaine, il faut environ 4 ans pour avoir un système à peu près complet. Au bout de ces 4 ans, les enchères ont fait des progrès et on réexamine les séquences. Ce qui garanti des revenus réguliers aux auteurs du site et aide beaucoup de bridgeurs de compétition.

Mais pourquoi attendre la troisième ou la quatrième enchère ? Pourquoi ne pas appliquer la stratégie des enchères dès la première enchère ?

Si on applique la stratégie des enchères dès la première enchère, on va avoir des séquences d'enchères plus courtes. Une séquence plus courte a trois avantage. Elle est plus simple. Elle en dévoile moins aux adversaires, rendant le flanc plus difficile. Et elle laisse moins de place pour intervenir.

\section{La stratégie des enchères}

Il y a deux types de séquences d'enchères. Les séquences compétitives et les séquences à deux.

Les séquences compétitives peuvent être symétriques, les deux camps ont entre 18 et 22H ou elles peuvent être dissymétriques avec un camp en attaque et un camp en défense.

Dans les séquences compétitives symétriques, il faut trouver son fit le plus vite possible. Évaluer la palier de «sécurité» distributionnelle et atteindre celui-ci en un minimum d'enchères. Il faut essayer de faire en sorte que ce soit l'adversaire qui prenne la dernière décision. En effet, celui qui prend la dernière décision peut se tromper.

Dans les séquences compétitives dissymétriques, il y a une difficulté supplémentaire. Il faut évaluer si on est en zone de manche, auquel cas, il faut la nommer même au delà du palier de «sécurité».

Dans les séquences à deux, la stratégie est plus compliquée. Il faut gérer à la fois la force et la distribution. Il faut évaluer la force combinée des deux mains pour savoir si on est en situation de partielle de manche ou de chelem.

En situation de partielle, il faut trouver un fit le plus vite possible, de préférence en majeure. En absence de fit, il faut jouer 1\NT ou en 5-2 au niveau de 2. Jouer 2\NT reste une possibilité faute de mieux.

En situation de manche, il faut trouver un fit en majeure. A défaut, il faut vérifier que 3\NT est jouable. Pour le savoir, il faut identifier que chaque joueur possède un solide arrêt dans le singleton du partenaire. Sans singleton on joue 3\NT. Si 3\NT est injouable, on tente une manche en mineure, ou à défaut dans un solide (avec beaucoup d'honneurs) semi-fit majeur.

En situation de chelem, il faut trouver son meilleur fit ou à défaut jouer 6\NT.

\section{La logique du Trèfle Rouge}

En Trèfle Rouge, on annonce toujours les majeures avant les mineures, même lorsque ces dernières sont plus longues.

On a donc besoin de 4 enchères pour décrire les majeures. On ouvre de 1\T avec 4 cartes à \C, on ouvre de 1\K avec 4 cartes à \P, on ouvre de 1\P avec 5 cartes à \P et de 2\C avec 5 cartes à \C. Ainsi, on trouve les fits 4-4 et les fits 5-3 au premier tour d'enchère.

Sans majeure, on ouvre de 1\NT, de 2\T ou de 2\K qui sont des enchères naturelles entre 12 et 14H et on ouvre de 1\C avec les jeux forts.

Ainsi, la logique du trèfle rouge est en accord avec la stratégie des enchères, trouver en priorité un fit majeur.

On ne nommera une mineure longue que pour une des trois raisons suivantes : Pour trouver la meilleure partielle et la mineure sera alors annoncée en Canapé. Pour éviter un mauvais 3\NT à cause d'un singleton non gardé. Et enfin pour jouer éventuellement un chelem.

Lorsqu'on trouve un fit mineur au niveau de 3, l'enchère qui suit est soit 3\NT, soit l'annonce d'un singleton, soit le déclenchement des enchères de chelem.

Voila, c’est tout.

Juste un dernier détail. On ne dispose plus d'enchère forte à 2\T ou 2\K. En conséquence, le trèfle rouge utilise la logique des trèfles forts dans une version plus moderne. Il y a deux groupes d'ouvertures. D'une part les ouvertures limitées 1\P, 2\T, 2\K et 2\C. L'ouverture 1\P, plus économique, est limitée à 17H. Les autres, au niveau de 2, sont très précises, elle vont de 12 à 14H. Ici encore, on aligne la logique du  système sur la stratégie des enchères : plus une enchère est chère, moins elle doit être fréquente. Et d'autre part, les ouvertures forcing 1\T, 1\K et 1\C qui contiennent en leur sein les mains forcing de manche, respectivement avec des \C, avec des \P et sans majeures.

Le traitement différent des \C et des \P peut donner une impression de complexité accrue. En réalité, il en est de même en majeure cinquième. Les \C et les \P posent des problèmes différents. A ce titre, il suffit de se rappeler de la fameuse séquence 1\C--1\P--2\C.

Un dernier détail. Quand on lit les réponses aux question d'Alain Levy sur le site «Bridge Academy» ou qu'on écoute les vidéos de Marc Kerlero sur son site «Amour du Bridge», on constate que ces deux auteurs, qui sont des champions pour gérer les différentes zones d'ombre de la majeure cinquième, utilisent une solution récurrente pour gérer les situations délicates. Cette solution universelle, connue de tous, est le Texas ou plus précisément le \textit{Puppet}, c'est à dire un Texas nébuleux suivi d'une clarification. Typiquement ces auteurs propose de jouer Texas ou \textit{Puppet} les séquences suivantes 1\P--1\NT--2\T--2\K (garantissant 5 cartes à \C pour l'un (Texas) et garantissant 5 cartes à \C ou 10H pour l'autre (Puppet)), 1\T--1\P--2\NT--3\C (pour pouvoir s'arrêter à 3\P d'une part et quand même pouvoir prospecter un chelem d'autre part). Et ce ne sont que les exemples les plus utiles.

En trèfle rouge, en réponses à l'ouverture, on utilise beaucoup de réponses en Texas. Et -- miracle ? -- cela permet de nettoyer beaucoup de séquences.

Au final, j'ai essayé de faire en sorte que le système soit le plus simple possible. Je laisse le soin à d'autres, le cas échéant, d'ajouter leurs gadgets pour en améliorer l'efficacité.


\section{Les failles de la majeure cinquième}

La majeure cinquième est un système efficace et surtout bien rodé par un demi-siècle d'utilisation intensive. Depuis son adoption comme standard français, deux grandes avancées théoriques ont modifié la stratégie des enchères. La première, c'est la loi des levées totales. Cette-ci ne concerne que les enchères compétitives et, en ce sens, a eu peu d'impact sur le cœur du système. La deuxième, c'est la théorie du singleton qui facilite grandement le choix entre une manche à \NT et une manche mineure. Le SEF 2024 ne prend en compte que partiellement cette deuxième avancée. Dans certaines situations, on annonce les forces et dans d'autres les singletons.

La première zone d'ombre de la majeure cinquième est son ouverture de 2\T. Je cite un champion : «Mon gars, c'est simple ! Si tu hésites entre ouvrir de 2\T et autre chose, choisi autre chose.» La messe est dite. En TPP, les ouvertures fortes sont rares donc leur inefficacité n'est pas vraiment coûteuse.

La séquence la plus archaïque et la plus coûteuse de la majeure cinquième est à mon sens
1\P--2\T--2\K--4\P. Même un défenseur médiocre va s'accrocher à sont 10 de carreau quatrième quand son partenaire a le Valet second. La notion même de soutien différé est un héritage d'une époque ancienne où le changement de couleur était la seule option forcing à disposition. En trèfle rouge, il n'y a pas de soutien différé. A partir de 11DH, on donne le soutien en Texas. Et ensuite, si on est dans la zone du chelem, et seulement dans ce cas, on montre sa longue utile. La séquence devient 1\P--2\C(*)--2\P--4\P ou 1\P--2\C(*)--2\NT--3\T à suivre. L'enchère de 2\C est un fit en Texas. L'ouvreur répond 2\P mini (12-14 points) ou 2\NT maxi (15-17 points). Notons au passage que la séquence 1\P--2\C(*)--2\NT--3\T est plus économique que son équivalent SEF 1\P--2\T--2\K--3\P, ce qui laisse toute la place disponible pour les amateurs de gadgets. De plus, l'ouvreur est déjà zoné, il ne lui reste plus qu'à dire les si les trèfles l'intéressent ou pas.

Exit les soutiens différés. Bienvenu aux fit Texas.

Une autre séquence coûteuse du SEF 2024 est la suivante 1\K--1\C--2\K--\Pass. Cette séquence, en apparence anodine, est médiocre. Sur l'ouverture de 1\K, avec son petit doubleton \C et 14H, l'adversaire a passé tranquillement en attendant de voir ce qui se passe. Au deuxième tour, il contre vaillamment, laissant le choix à son partenaire entre les \P et les \T. En trèfle rouge, la séquence est plus rapide 2\K--\Pass, ce qui laisse les adversaires dans une situation bien inconfortable.

Pourquoi la séquence ci-dessus est-elle médiocre ? En ouvrant de 1\K, on demande au partenaire de nommer une majeure que de toute façon on ne va pas soutenir. On provoque ainsi une enchère inutile (et toute enchère inutile est nuisible, inutile pour nous, utile pour l'adversaire) à 1\C.

Je ne sais pas si j'ai réussi mais ce que j'ai essayé de faire, c'est de nettoyer au maximum le système de toutes ces enchères inutiles.

\section{Un peu d'histoire}

Pourquoi ce nom, le Trèfle Rouge, et sa version anglaise Red Club. Il y a une raison mnémotechnique, l'ouverture de 1\T promet 4 cartes à \C qui est une couleur rouge. Et une raison historique, c'est un hommage au Trèfle Bleu, Le Blue Team Club une référence absolue en matière de trèfle fort. On peut télécharger la traduction anglaise de ce système italien sur le site «Clairebridge». Cet excellent système est daté, ne prenant pas en compte le principe de la majeure par cing. Il a tout de même permis au italiens de dominer la scène mondiale pendant plus de dix ans.

Les systèmes de trèfle fort ont un avantage compétitif. Ils permettent d'améliorer l'efficacité de toutes les autres ouvertures. Sans trop sacrifier les mains fortes puisque l'ouverture de 1\T, économique laisse toute la place nécessaire au développement de la main. En incorporant le principe de la majeure cinquième, les trèfles forts prennent le nom de trèfle de précision et sont toujours en vogue.

Il y a une parade aux systèmes de trèfle fort. Quand vos adversaires ouvrent de 1\T, il n'y a pas de manche à jouer dans votre camp. Vous pouvez donc intervenir de façon totalement anarchique. Cela prive l'ouvreur des paliers nécessaires pour développer son système sans vraiment les renseigner pour le jeu de la carte.

C'est cette parade qui rend impopulaire les systèmes de trèfles forts. En effet, vous passer des heures à apprendre des séquences de relai. Enfin l'ouverture chérie de 1\T arrive et vous vous apprêter à montrer à votre adversaire ce qu'est du bon bridge et celui-ci, goguenard , intervient à 1\P dans le 10 quatrième. Quelle frustration !

Pour parer à cette parade, on a développé deux astuces. Il y a le trèfle polonais. C'est un système proche du trèfle fort mais avec une ouverture de 1\NT fort. De sorte que l'ouverture de 1\T est soit 12-14H comme d'habitude, soit forte. Comme l'ouverture banale est légèrement plus fréquente que l'ouverture forte, intervenir de façon anarchique devient très risqué. La deuxième idée, c'est le système de Fantoni et Nunes. Toutes les enchères au niveau de un sont forcing et potentiellement forte et les enchères au niveau de deux limitées. De sorte que , fort ou faible, on commence toujours par nommer sa couleur.
Le prix à payer, en terme de complexité, de ce dernier système, est élevé. Les ouvertures au palier de deux, faute de place, sont assorties d'un système de relai vraiment complexe.

J'ai un peu mixer ces deux idées. En trèfles rouge, les ouvertures de 1\T et de 1\K comportent un peu plus de mains banales que de mains fortes. Quant à l'ouverture de 1\C, quand bien même promet-elle 15H, elle n'exclue pas une manche majeure chez les adversaires. Dans les deux cas, il serait contre productif pour les adversaires d'intervenir de façon anarchique. Et d'autre part, en cas d'intervention, on a déjà commencé à décrire sa main, tout particulièrement en ce qui concerne les majeures.

En résumé, dans mon esprit, le Trèfle Rouge est un descendant du Trèfle Bleu.














\chapter*{Exemples}
{
\rowcolors{0}{bananamania}{bananamania}
\setdefaults{bidlong=on}

Voici le résultat du test d'un tournoi de club.

\begin{enumerate}

 \item Ouverture de 1\NT fort

\hand!{A76}{AQJ9}{AT2}{QT5} \quad \hand!{T85}{853}{Q5}{KJ642} \quad
\begin{biddingpair}
 1C & 1D \\
 1N \\
\end{biddingpair}

L'ouverture de 1\T promet 4 cartes à \C. Le relai à 1\K montre de 5 à 11H sans 4 cartes à \P. Le redemande à 1\NT équivaut à une ouverture de 1\NT fort.

L'adoption du \NT faible n'est pas sans inconvénients. Ici, l’adversaire a eu deux occasions de nommer les piques.

\item Bicolore

\hand!{AKT52}{}{K85}{AJ765} \quad \hand!{J}{A432}{JT97}{9843} \quad
\begin{biddingpair}
 1S & 1N \\
 3C \\
\end{biddingpair}

L'ouverture de 1\P est commune à presque tous les systèmes existants. Avec son singleton, Est va à la pêche au bicolore en utilisant le relai forcing de 1\NT. L'enchère de 3\T indique un bicolore au moins 5-5 dans une ouverture maximum.

Au box-office, les paires jouant 3\T+1 se sont vues attribuées la note de 90,59\%. On a sans doute inhibé un réveil à \C en enchérissant ainsi.

\item Misfit

\hand!{AT3}{AKJT82}{2}{432} \quad \hand!{J}{5}{AKJ9653}{JT85} \quad
\begin{bidding}%
  2H & 2S & X & 3S \\
  p & p & p \\
\end{bidding}

En ouvrant de 2\C, Ouest a relativement bien vendu sa main. Malgré sa sixième carte à \C, il parait prudent d'attendre le deuxième Contre pour annoncer 4\C. Bonne pioche.

Au box-office, 4\C-2 vaut 40.59\%. Si la défense s'avise de donner une levée, la note monte à 68,82\%. Faire chuter 3 \P ne demande pas un gros effort et rapporte 84,12\%.

Sur ouverture de 1\C et une intervention à 2\P, beaucoup de joueurs en Est se sont sentis des ailes et ont préférés l'enchère forcing de 3\K au contre d'appel. Après cela, la catastrophe était inévitable. Sur une ouverture de 2\C, limitée à 14H, il est plus facile pour Est de juger correctement sa main.

\item  Sans-atout faible

\hand!{AJT82}{AQ6}{K53}{T6} \quad \hand!{Q73}{K}{AJ84}{AJ972} \quad
\begin{biddingpair}
 1N & 2C\\
 2S & 4S\\
\end{biddingpair}

L'ouverture de 1\NT promet un jeu régulier de 12 à 14H. Cette ouverture peut comporter une majeure cinquième mais pas une majeure quatrième.
Le répondant interroge avec un Stayman. Quand il apprend que l'ouvreur possède 5 cartes à \P, il peut conclure.

Cette séquence est meilleure que la séquence 1\P--2\T--2\P--3\P--4\P produite à d'autres table. Cela dit, sur cette donne particulière, le flanc peut dormir tranquille.

\item Sans-atout fort

\hand!{KQ9}{Q72}{AKJ95}{74} \quad \hand!{J72}{AT94}{Q62}{KT2} \quad
\begin{biddingpair}
 1H & 1S \\
 1N & 3N\\
\end{biddingpair}

Cette séquence est équivalente à la séquence 1\NT--2\T--2\K--3\NT du SEF. Seuls les adeptes du double Stayman s'en sorte mieux avec la séquence
1\NT--2\NT--3\T--3\P--3\NT qui dévoile un peu moins la main de l'ouvreur.

Les trèfles étaient 6-2 et l'entame trèfle faisait chuter le contrat de deux levées pour un maigre 37,65\%. Quelques heureux s'en sortirent avec une de chute.

\item Sans-atout faible

\hand!{AQ93}{K64}{AT95}{75} \quad \hand!{4}{QJ75}{KJ32}{A864}\quad
\begin{biddingpair}
 1D & 1H \\
 1S & 1N \\
\end{biddingpair}

Avec 4 cartes à pique, il est interdit d'ouvrir de 1\NT. Ou ouvre de 1\K, enchère dont la fonction, justement, est de montrer ces 4 cartes. Le relai à 1\C montre de 5 à 11H. Pas d'espoir de manche donc. La redemande à 1\NT montrerait une ouverture de 1\NT fort. Avec une ouverture de 1\NT faible, on les annonce en Texas. L'enchère de 1\P est Puppet pour 1\NT.

Cette séquence est supérieure à la séquence 1\K--1\C--1\P--1\NT qui a malheureusement dévoilée les 4 cartes à \C du déclarant et, mais ça ne joue que pour l'entame, les 4 carreaux de l'ouvreur.

La feuille de route est très folklorique. Les déclarants font entre 7 et 10 levées quand ils jouent à \NT.



 \item Sans-atout fort

 \hand!{KT86}{KQ8}{73}{AQJ9} \quad \hand!{J7532}{AJ95}{AJT}{7} \quad
 \begin{biddingpair}
  1D & 2H \\
  2N & 4C \\
  4S \\
 \end{biddingpair}

  Ouest ouvre de 1\K pour montrer ses 4 cartes à \P. Est le soutien à 2\C montrant une main au moins limite de manche.
  Ouest ne peut pas se contenter de rectifier à 2\P (non forcing) qui montrerait une main faible. Sur 2\NT, le splinter est une petite tentative de chelem que l'ouvreur refuse immédiatement avec deux petits honneurs à \T. De plus, il a déjà montré sa force.

  Au final, la main du déclarant a été moins dévoilée qu'après une ouverture de 1\NT fort suivi d'un Stayman.


 \item Jeu fort à \C

 \hand!{KQ}{AQJT6}{Q2}{AJT5} \quad \hand!{9732}{3}{A98754}{87}\quad
 \begin{biddingpair}
  1C & 1D \\
  1H & 1S \\
  2N & \\
 \end{biddingpair}

 A partie de 15H, on ne peut pas ouvrir de 2\C, on ouvre de 1\T. Avec ce bel As, la réponse de 1\P serait assez pessimiste. Ouest hésite à redemander 2\C, forcing de manche mais il n'a que 5 cartes et il lui manque un point. L'enchère de 1\C est forcing de toute façon. La redemande à 1\P tire la sonnette d'alarme. Ouest se contente donc de proposer la manche et Est passe.

 Selon son talent, le déclarant peut gagner (62\%) ou chuter (25\%).

 \item Jeu fort indéterminé

 \hand!{AJ}{4}{AQ954}{AK973}\quad\hand!{T93}{AKQ9852}{T7}{J}\quad
 \begin{biddingpair}
  1H & 2D \\
  2S & 3H\\
  4H & \\
 \end{biddingpair}

 L'ouverture de 1\C promet 15H sans majeure quatrième. La réponse de 2\K est un Texas \C. Sans enchère naturelle, l'ouvreur utilise l'enchère de 2\P comme Joker pour imposer une manche quelque part. Est propose un chelem à \C mais Ouest a déjà bien vendu sa soupe.

 \item Bicolore majeure

 \hand!{AQJ93}{AKT974}{}{84}\quad\hand!{742}{62}{KQ92}{AQJ6}\quad
 \begin{biddingpair}
  1C & 1S \\
  2H  & 2N \\
  3S & 4H \\
 \end{biddingpair}\quad
 \begin{biddingpair}
  1S & 2N \\
  4S&\\
 \end{biddingpair}

 Normalement, il faudrait 15H pour ouvrir de 1\T mais Ouest a jugé préférable d'ouvrir de 1\T car il maîtrise mieux la séquence.
 Beaucoup de sueur dans cette séquence.
 Cela dit s'il avait ouvert de 1\P, tout aurait été plus rapide !

 Sur ouverture de 1\P, Est fait un signal, soit de faiblesse, soit de platitude, en disant 1\P. Il a une main banale de manche. Ouest montre 5 cartes à \C.
 En reparlant, Est impose la manche. Conclure à 3\NT serait , hâtif, Ouest pourrait très bien avoir singleton \P. Cela permet à Ouest de montrer sa distribution exceptionnelle et de récupérer le fit in extremis.

 \item Contre de l'ouverture forte

 \hand!{A98}{QJ2}{KJ63}{KJ6}\quad\hand!{42}{43}{AQT97}{AT83}\quad
 \begin{bidding}
  1H & X & {\Redouble!} & 1S \\
  p & p & 2S & p \\
  2N & p & 3N & \\
 \end{bidding}

 \item Grand chelem

 \hand!{A2}{A}{KT}{AKQ97642}\quad\hand!{8654}{KT9}{A954}{T8}\quad
 \begin{biddingpair}
  1H & 1N \\
  3C & 3D\\
  4C & 4H\\
  7N & \\
 \end{biddingpair}

 Sur ouverture de 1\C, l'enchère de 1\NT indique 6-8H. Le saut à 3\T est forcing de manche. Est ne sait pas très bien où il en est et se contente de montrer qu'il ne craint pas le singleton carreau. Sur 4\T, la situation est plus claire. En matière de chelem mineur, Est-Ouest jouent le Turbo et 4\C montre le contrôle de la couleur et une carte clef. La conclusion est évidente et vaut 96\% au box-office.

 \item Une main banale pour finir (le reste des donnes est joué en Nord-Sud)

 \hand!{AJ95}{AKT64}{K4}{AQ}\quad\hand!{QT6}{J532}{Q8}{J974}\quad
 \begin{biddingpair}
  1C & 1S \\
  2H & 3H \\
  4H & \\
 \end{biddingpair}

 L'enchère de 1\P est une alerte jeu faible. C'est le seul cas où le soutien peut être différé en Trèfle Rouge. Ouest n'a plus les moyens d'imposer la manche et se contente de monter 5 cartes à \C. Est, finalement, est moins nul qu'on aurait pu le croire et fait un effort. Ouest récompense cet effort.



  \end{enumerate}}

\chapter{Les ouvertures}
bboalert,RCbase.tex

\begin{multicols}{2}


\section*{Les  unicolores}


Attention, les mains 5-3-3-2 sont toujours considérées comme des mains régulières, même avec une majeure cinquième.
Les unicolores mineurs 6-3-2-2 de 12 à 14H s'ouvrent de 1SA.

Avec les piques, on ouvre de 1\P de 11H à 17H et de 1\K à partir de 18H. L'ouverture de 1\P est la seule ouverture agressive du système.
Avec un bicolore 5-5 à honneurs concentrés, on peut ouvrir avec 10 points d'honneur seulement.

Avec 6 cartes à \P, on ouvre de 2\P jusqu'à 10H et de 1\P à partir de  11H. Il n'y a pas de zone grise où on est trop fort pour ouvrir de 2\P et trop faible pour ouvrir de 1\P. En fonction de la position et de la vulnérabilité, la limite basse de l'ouverture de 2\P est élastique.

Avec les cœurs, on ouvre de 2\C de 12H à 14H et de 1\T au delà.

Avec les trèfles, on ouvre de 2\T de 12H à 14H et de 1\C au delà.


Avec les carreaux, on ouvre de 2\K de 12H à 14H et de 1\C au delà.

\section*{L'ouverture de 2SA}

Les mains régulières de 20 ou 21H s'ouvrent de 2SA. Cette ouverture est prioritaire même avec une majeure 4\ieme ou 5\ieme.

\section*{Les bicolores majeurs}

Dans la première zone de l'ouverture, de 12 à 14H, on annonce toujours les piques avant les cœurs. Donc avec 4 cartes à \P et 5 cartes à \C, on ouvre de 1\K.

Avec un jeu fort, 15H et plus, on annonce la couleur la plus longue en premier. On ouvre donc de 1\T avec les \C et de 1\P (ou de 1\K avec les jeux très forts, 18H+, avec les piques.)
En cas d'égalité, 5-5, ou 6-6, on nomme les \P en premier.

Avec un beau bicolore majeur, sans point perdu, on peut ouvrir à 11H et parfois même à 10H avec un beau bicolore 5-5.


\section*{Les bicolores mineurs}

Les mains 5-4-2-2 sont systématiquement assimilées à des mains régulières et n'entrent pas dans la cadre de ce paragraphe.
A partir de 15H, les bicolores mineurs s'ouvrent de 1\C.
De 12 à 14H, dans une main 5-4-3-1 ou 6-5, on ouvre dans la couleur la plus longue au niveau de 2.
Avec un bicolore 5-5 ou 6-6, on ouvre de 2\K.

\section*{Mixte majeur-mineur}

Qu'elle soit quatrième ou cinquième, \textbf{on nomme toujours la majeure en premier.}
Avec 5 cartes à cœur, une mineure quatrième et deux doubletons incluant des honneurs, dans une main de 12H à 14H, on pourra choisir d'ouvrir la main de 1SA.

\section*{Les mains régulières}

Les mains 4-4-4-1 suivent la même logique que les mains régulières.

Dans la zone 12-14H, on ouvre de 1\NT sans majeure 4\ieme (mais possiblement avec une majeure 5\ieme), on ouvre de 1\T suivi d'une redemande à 1\P avec 4 cartes à \C (et éventuellement 4 cartes à \P) et on ouvre de 1\K suivi d'une redemande à 1\P avec 4 cartes à \P sans 4 cartes à \C.

Dans la zone 15-17H, on ouvre de 1\C sans majeure 4\ieme, on ouvre de 1\T suivi d'une redemande à 1\NT avec 4 ou 5 cartes à \C et on ouvre de 1\K suivi d'une redemande à 1\NT avec 4 ou 5 cartes à \P.

Dans la zone 18-19H, on ouvre de 1\T, 1\K ou 1\C selon les majeures. La redemande est variable. On redemande 1\NT si le répondant a utilisé le signal de faiblesse à 1\P. Sur le relai ordinaire (1\K sur 1\T ou 1\C sur 1\K), on nomme la majeure au niveau de un avant de faire la redemande à 2\NT.

Dans la zone 20-21H, on ouvre de 2\NT.

Au dela de 22H, on ouvre au niveau de un et on fera une redemande à saut (forcing de manche) sauf sur le signal de faiblesse à 1\P qui décale la zone forcing de manche à 24HL.
\end{multicols}


 \enchbox{Tableau des ouvertures}{
 1\T & 12H+ & 4+ cartes à \C (\T ou \K possiblement plus long)\\
 1\K & 12H+ & 4+ cartes à \P(\T \K ou \C possiblement plus long)\\
 1\C & 15H+ & pas de majeure 4\ieme\\
 1\P & 11-17H & 5 cartes à \P\\
 1\NT & 12-14H & régulier\\
 2\T & 12-14H & 5+ cartes à \T\\
 2\K & 12-14H & 5+ cartes à \K\\
 2\C & 12-14H & 5+ cartes à \C, 0-3 cartes à \P\\
 2\P & 0-10H & 6 cartes à \P\\
 2\NT & 20-21H & régulier\\
 (3|4)[\T\K\C\P] &5-10H& barrage constructif (sauf vert contre rouge)\\}





\chapter{L'ouverture de 1\T}



\begin{multicols}{2}


L'ouverture de 1\T indique au moins 4 cartes à \C et au moins 12H. L'enchère est forcing et illimitée.

La réponse la plus négative est 1\P, toutes les autres réponses promettent 5H ou un As. Cette réponse de 1\P couvre deux cas, les mains très faibles et les mains forcing de manche sans autre enchère à disposition.

Il n'y a pas de notion de fit différé (à l'exception des mains très faibles). Si le répondant est fitté à \C, il doit le dire immédiatement. Avec un jeu banal pour jouer la manche, il répond 2\NT et l'ouvreur avisera. Avec une main limite de manche ou une main de chelem, il utilise le Texas à 2\K. Avec un jeu faible, mais constructif, il donne un soutien direct. Avec un jeu vraiment faible, le répondant commence par faire l'enchère très négative de 1\P.

Attention aux splinters. Les splinters en face d'une ouverture de 4 cartes sont plus forts que les splinters en face d'une ouverture de 5 cartes. On n'est pas en sécurité distributionnelle.

Sans fit, et jusqu'à 11H, le répondant répond 1\K sans 4 cartes à \P et 1\C sinon. A partir de 12H, il nomme sa couleur en Texas ou commence par le relai à 1\P s'il n'a pas de couleur intéressante. Attention, comme dans tout le reste du système, nommer une majeure quatrième est prioritaire même en présence d'une mineure plus longue.

Les changements de couleur à saut permettent de décrire en une seule enchère des unicolore limite de manche. Très pratique. On vend la main en une seule enchère.

L'ouverture de 1\T est à la fois la plus fréquente et la plus complexe.
\end{multicols}



\enchbox{Ouverture 1\T}
{
 1\K & 5-11H & Pas de majeure 4\ieme\\

 1\C & 5H+ & Au moins 4 cartes à \P\\

1\P & 0-4H & Toute distribution\\
  \rb & 5-7DH & 4+ cartes à \C \\
  & 12H+ & Ni majeure 4\ieme ni mineure 5\ieme\\
1\NT & 12H+ & 5 cartes à \T\\

 2\T & 12H+ & 5 cartes à \K\\

 2\K & 11DH+ & 4+ cartes à \C\\


 2\C& 8-10DH & 4+cartes à \C\\
  2\P & 9-11H &  6+ cartes à \P \\
 2\NT & 12-15H& 4(5) cartes à \C \\
 3\T  & 9-11H &  6+ cartes à \T \\

 3\K & 9-11H &  6+ cartes à \K \\
3\C &7-9H& 5 cartes à \C \\

   3\P &8-11H& Splinter chicane indéterminée\\
  3\NT &8-11H & Splinter singleton \P\\
  4\T &8-11H& Splinter singleton \T \\
 4\K &8-11H& Splinter singleton \K\\
 4\C && Barrage 6 cartes\\
  4\P & 0-9H & Naturel 7+ cartes\\
  4\NT && Blackwood brutal\\
}

%%%%%%%%%%%%%%%%%%%%%%%%%%%%%%%%%%%%%%%%%%%%%%%%%%%%%%%%%%%%%%%%%%%%%%%%%%%%%%%%%


\titre{
  1\T
  Réponses après intervention
}

\begin{multicols}{2}


En cas d'intervention, on conserve l'esprit du système. Toutefois, les Texas mineurs perdent leur caractère forcing de manche et peuvent se faire dès 8H avec une belle couleur.
Le seul schéma à bien mémoriser, c'est celui du contre d'appel. Comme l'enchère est fréquente, cela devrait se faire assez vite.


\section*{Intervention par \Double}

Le \Redouble est une enchère punitive. Il dénie 4 cartes à \C et promet 10H et deux couleurs 4\ieme. Tous les contre subséquents sont punitifs.
Alternativement, en fonction de la vulnérabilité, on peut préférer annoncer 3\NT ou faire un relai à 1\P pour faire jouer les \NT de la main de l'ouvreur. L'enchère de 1\P devenant exclusivement forcing de manche.

Avec 4 cartes à \C et une main de manche, on conserve la réponse à 2\NT.

Avec une main unicolore, à partir de 8H, on utilise une enchère Texas 1\NT ou 2\T. L'enchère Texas de 1\C promet 5 cartes à \P et non plus 4.

Avec une main fittée limite de manche ou ambition de chelem, on passe par le Texas à 2\K.
L'enchère de 2\C garde sa signification fittée 8-10 DH.

L'enchère de 1\K est artificielle. Elle promet au moins 7-8H et sert d'enchère poubelle. Pas envie de punir, pas envie de laisser jouer, pas de fit 4\ieme.
Une mineure cinquième de mauvaise qualité par exemple ou 4 cartes à pique. Le plus souvent, on va jouer à \NT mais on laisse la place au déclarant pour déclarer 5 cartes à \C le cas échéant. On laisse aussi un peu de place aux adversaires pour s'exprimer, ça sera utile pour le jeu de la carte.


\section*{Intervention par 1\K}
Lorsque l'adversaire intervient naturellement, on va apporter quelques modifications. D'une part le \Double remplace l'enchère poubelle de 1\K, à ceci près qu'elle promet 8H.
L'enchère Texas de 1\C n'est pas modifiée ni sa suite. L'enchère de 1\P devient exclusivement forcing de manche. Les enchères Texas de 1\NT et de 2\T ne promettent plus que 8H.
le répondant ne s'engage à reparler que s'il à 12H et plus. Un ouvreur dotté de 14H ou plus ne pourra pas se contenter d'une simple rectification. A partir de 2\K, le système est inchangé.


\section*{Intervention naturelle par 1\C}
le \Double est d'appel et recherche en priorité un fit à \P. L'enchère de 1\P devient exclusivement forcing de manche. Les enchères Texas de 1\NT et de 2\T ne promettent plus que 8H.
Les enchères de 2\K et de 2\C sont inchangées, on part du principe que les adversaires se sont trompé.

\section*{Intervention artificielle non bicolore par 1\C}
Que l'enchère soit naturelle ou artificielle, on ne change rien. il est impossible de prévoir ce que les adversaires peuvent inventer.
Si l'enchère de 1\C promet explicitement une longueur à \P, il est clair que le \Double est plutôt à la recherche d'une mineure\dots

\section*{Intervention par 1\P}
Le \Double promet 8H et n'est plus limité à 11H puisqu'on ne dispose plus de l'enchère de 1\P. Les enchères Texas de 1\NT et de 2\T ne promettent plus que 8H.
L'enchère de 2\K montre une main limite de manche ou à la recherche d'un chelem. Le cue-bid indique un jeu de manche sans, a priori, d'espoir de chelem.

\section*{Intervention par 1\NT}

Le contre est punitif, à partir de (9)10H. L'enchère nous prive de Texas. Toutes les enchères au niveau de 2 sont naturelles non forcing. Il est possible de rater un fit à \P.

La seule difficulté est la suite à donner après un contre. Les adversaires peuvent continuer de façon naturelle ou en Texas. Il convient d'être précis sur le caractère punitif du \Double.
La règle utilisée ici est la suivante : «\textit{Le contre de toute enchère nommée naturellement est punitif}»

Lorsque les adversaires sauvent le contrat, il faut être clair sur les priorités : Avec 5 cartes à \C, l'ouvreur doit en priorité rechercher un fit 5-3. Avec 4 cartes à \P, il faut soit punir avec 4 cartes dans leur couleur, soit rechercher le fit \P. Avec 3 cartes à \P et en acescence de fit \C, il arrivera parfois de devoir contrer punitivement en réveil avec seulement 3 atouts. La vulnérabilité a beaucoup d'importance quant à la conduite à tenir ici.



\enchbox{1\T - 1\NT - \Double - 2\T (Texas)}
{
 \Pass & Banal Ou punitif à \K\\
 \Double & Appel \\
 2\K & Pas envie de punir\\
 2\C & 5 cartes 15-17H\\
 2\P & Bicolore cher\\
 3\T & Canapé

}

Si l'enchère est naturelle

\enchbox{1\T - 1SA - \Double - 2\T (Naturel)}
{
 \Pass & Appel\\
 \Double & Punitif\\
 2\K & Canapé\\
 2\C & 5 cartes 15-17H\\
 2\P & Bicolore cher\\
 3\T & Pas envie de punir
}



\section*{Intervention 2\T ou 2\K}

A partir de 2\T, on oublie les Texas. Le \Double est Spoutnik et promet 4 cartes à \P.
L'enchère de 2\K devient naturelle, forcing un tour, de même que l'enchère de 2\P.
Et l'enchère de 3\C indique un soutien limite de manche.


\section*{Intervention bicolore}
Les enchères naturelles sont non forcing. Les cue-bids sont forcing de manche. le cue-bid économique est fitté et le deuxième cue-bid indique 5+ cartes dans la quatrième couleur.

\section*{Réveil de l'ouvreur}
Court dans la couleur adverse, le \Double vient bien.
Avec un Canapé, on le nomme.

\end{multicols}


%\section*{Attitude de l'ouvreur}

%%%%%%%%%%%%%%%%%%%%%%%%%%%%%%%%%%%%%%%%%%%%%%%%%%%%%%%%%%%%%%%%


\titre{
  1\T -- 1\K}


Le répondant a montré un jeu de 5 à 11H.
Les enchères à saut (sauf 2\NT) sont forcing de manche.

La redemande de 1\NT indique une main de 15-17H. Avec une main régulière de 12 à 14H ou de 22H+, on redemande 1\P, enchère Texas pour 1\NT. Cette façon de faire permet, le plus souvent de faire jouer la main du même coté que le champ, aussi bien sur l'ouverture de 1\T que sur celle de 1\K.

Par ailleurs, la redemande à 1\P permet aussi d'annoncer les jeux forts non forcing de manche.

\textit{Avec un jeu régulier de 15-17H et 5 cartes à \C, il y a débat. Faut-il répondre 1\C ou 1\NT ? Dans l'état actuel du système, on répond 1\NT de sorte que 1\C montre un jeu excentré ou de 18H+}

\enchbox{1\T -- 1\K}
{
 1\C & 15H+ & 5+ cartes à \C\\
 1\P & 12H-14H & Régulier\\
 \rw & 15H+ & Excentré, exactement 4 cartes à \C \\
 & 18H+ & Exactement 4 cartes à \C\\
 1\NT & 15-17H & Régulier\\
 2\T & 12-14H & Canapé\\
 2\K & 12-14H & Canapé\\
 2\C & 20H+ & 5+ cartes à \C Forcing de manche\\
 2\P & 20H+ & 4 cartes à \C Forcing de manche\\
 2\NT & 18-19H & Régulier 4 cartes à \C\\
 3\T &20H+& Naturel Forcing de manche\\
 3\K &20H+& Naturel Forcing de manche\\
 3\C &20H+ & 6+ cartes à \C Forcing de manche\\
 3\P & 18H+ & 6-5 forcing de manche \\
}

%%%%%%%%%%%%%%%%%%%%%%%%%%%%%%%%%%%%%%%%%%%%%%%%%%%%%%%%%%%%%%%%

\titre{1\T--1\K--1\C}

\begin{multicols}{2}


La redemande de 1\C indique au moins 5 cartes et va de 15 à 19H. La réponse de 1\K a fortement limité les possibilités de chelem. Et la manche reste incertaine. L'enchère est forcing un tour.

Avec un jeu faible, le répondant fait la redemande habituelle à 1\P. Il faut la même enchère à partir de 9H sans enchère naturelle.

Avec 3 cartes à \C, donner le fit est obligatoire (sauf 5-6H). Le répondant redemande alors 2\C, 3\C ou 4\C en fonction de sa force. Au delà de 2\C, la situation étant forcing de manche, c'est l'enchère de 3\C la plus forte.

\enchbox{1\T--1\K--1\C}
{
1\P & 5-6 ou 9+ & Faible ou banal\\
1\NT & 7-8H & Régulier \\
2\T  & 7-8H & 6 cartes \\
2\K  & 7-8H & 6 cartes \\
2\C  & 7-8DH & 3 cartes \\
3\C  & 11DH+ & 3 cartes \\
4\C  & 9-10DH & 3 cartes\\
}

Sur 1\P, l'ouvreur peut nommer les \NT, il peut faire un Canapé. Il peut même annoncer 2\P avec un 6-5 faible.

Sur 1SA, l'ouvreur doit faire un saut s'il veut imposer la manche.

\end{multicols}

\titre{1\T--1\K -- 1\P}

\begin{multicols}{2}

L'enchère de 1\P est un  Puppet pour 1\NT. Mais le répondant peut refuser de rectifier. Il y a deux situations où il va refuser.
Avec 11H, il rectifie à 2\NT au lieu de 1\NT pour ne pas rater la manche en face de 14H.

Avec une mineure longue, il la nomme. (La main ne peut pas être forte, sinon, il l'aurait nommé au tour précédent)

Le plus souvent, le répondant se contente de rectifier à 1\NT. Maintenant, en reparlant, l'ouvreur peut montrer un jeu fort. Tout est naturel.

\enchbox{1\T--1\K -- 1\P}
{
1\NT & 5-10H & Régulier\\
2\T & 5-10H & (5)6 cartes \\
2\K & 5-10H & (5)6 cartes \\
2\C & 5-9H & Bicolore mineur \\
2\P & 10-11H & Bicolore mineur \\
2\NT & 11H & Régulier\\
3\T & 5-8H & 7 cartes \\
3\K & 5-8H & 7 cartes \\
}

\enchbox{1\T--1\K -- 1\P--1\NT}
{
2\T & 15-17H & Canapé\\
2\K & 15-17H & Canapé \\
2\C & 18+H & 4-4-4-1\\
-> & 2\P\ & relai quasi-obligatoire\\
->->&& 2\NT singleton \P\\
->->&& 3\T singleton \T\\
->->&& 3\K singleton \K\\
2\P &  & Bicolore 6-5 \\
2\NT & 22H+ & Régulier \\
3\T & 18-19H & Canapé \\
3\K & 18-19H & Canapé \\
}

Sans enchère naturelle, l'ouvreur peut passer par 2\C. La main est 4-4-4-1 mais assez mal zonée. Le singleton peut être à pique ou en mineure. Le répondant interroge le singleton par un relai à 2\P.



\end{multicols}


\titre{1\T--1\K--1\NT}

On se retrouve avec l'équivalent d'une ouverture de 1\NT fort. A ceci près que le problème des fits majeurs est déjà résolu. Il n'y en a pas.

De plus, avec un répondant limité à 11H, on ne voit pas bien ce qu'on irait faire en mineure !

Le répondant peut passer, proposer la manche en disant 2\NT (ou 2\C avec 3 cartes) ou conclure à 3\NT.

Avec une belle mineure longue, le répondant l'aurait nommée au tour précédent. Les enchères de 2\T et 2\K sont donc logiquement non forcing. Avec les deux mineures et 9-11H, on peut utiliser l'enchère de 2\P. 

Avec un singleton dans une main de 10--11H, le répondant peut avoir envie d'éviter un mauvais 3\NT. Mise à part l'enchère de 3\C qui est une enchère de politesse, toute enchère au niveau de 3 montrent un singleton.


\titre{1\T--1\K--2\NT}

Sur toute redemande de l'ouvreur à 2\NT, 3\T est un check-back Stayman (inutile dans cette séquence particulière mais on ne veut pas créer d'exception).

Les autres enchères au niveau de 3 montrent un singleton (3\C dans la mejeur de l'ouvreur pour le singleton \T)


\titre{
  1\T -- 1\C}

Le répondant a montré 4 cartes à  \P.

L'enchère de 1\C dévore un peu d'espace. Cela a deux conséquences. Le relai à 1\P devient moins précis. Il montrera le plus souvent une main régulière de 12-14H non fitté à \P. mais il peut cacher un fit \P forcing de manche ainsi que tout un tas de mains fortes. D'autre part, la réponse de 1SA peut masquer 5 cartes à \C.

\enchbox{1\T -- 1\C}
{

 1\P & 12H+ & Relai fourre-tout\\
 1\NT & 15-17H  & Régulier\\
 2\T & 12-14H & Canapé\\
 2\K & 12-14H & Canapé\\
 2\C & 15-17H & 5 cartes à \C, irrégulier \\
 2\P & 12-14H & 4 cartes à \P\\
 2\NT & 18-19H& Régulier\\
 3\T &(18)19H+& Naturel Forcing de manche\\
 3\K &(18)19H+& Naturel Forcing de manche\\
 3\C & 19H+ & 6 cartes à \C Forcing de manche\\
 3\P & 17-19DH & 4 cartes à \P\\
}

\titre{1\T--1\C--1\P}

\begin{multicols}{2}



C'est la séquence la plus ambiguë.

Le fit \C est exclu mais le fit \P est encore possible. Si l'ouvreur a une main forte, il le fera savoir. Une redemande ultérieure à \P de l'ouvreur montrera une main très forte (au moins 20 points) et sera forcing de manche.

Avec 5 cartes à \P, le répondant répète ses piques avec un jeu faible ou passe par l'enchère de 2\C avec au moins 11H.

Les enchères en mineure sont des Canapés. Et tout ce qui est au niveau de 3 est forcing de manche.

\enchbox{1\T--1\C--1\P}
{
1\NT & 5-10H & 4 cartes à \P \\
2\T  & 5-10H & 5+ cartes à \T \\
2\K  & 5-10H & 5+ cartes à \K \\
2\C  & 11H + & 5+ cartes à \P \\
2\P  & 5-10H & 5+ cartes à \P \\
2\NT & 11H & Fourre-tout \\
3\T & 12H+ & 5+ cartes à \T \\
3\K & 12H+ & 5+ cartes à \K \\
3\C & 16H+ & Régulier \\
3\P & 15H+ & 6 belles cartes à \P \\
3SA & 13-15H & Régulier \\
}


\end{multicols}

\titre{1\T--1\C--1\P--1\NT}

\begin{multicols}{2}
 Le plus souvent, avec une main banale de 12-14H, l'ouvreur va passer. Toutefois, il peut avoir tout un panel de mains fortes, voire très fortes.
 Le principe est le suivant : toute enchère à saut est forcing de manche, soit dans une optique de chelem, soit dans une optique de recherche de la meilleure manche.
 Toute enchère sans saut indique la zone au dessus que la même enchère non précédée de 1\P.

 \enchbox{1T--1\C--1\P--1\NT}
 {
 2\T & 15-17H(18) & 5 cartes à \T \\
 2\K & 15-17H(18) & 5 cartes à \K \\
 2\C & 18-19H & 5 cartes à \C \\
 2\P & 18-19H & 4 cartes à \P \\
 2\NT & 22-23H & Régulier \\
 }

\end{multicols}

\titre{1\T--1\C--1\NT}

\begin{multicols}{2}
L'ouvreur possède 15-17H régulier avec 4 ou 5 cartes à \C. Et le répondant possède au moins 4 cartes à \P. Un fit reste donc possible dans chaque majeure.

De plus, la main du répondant est illimitée.

Une chose est sûre, l'enchère de 2\C est une réitération du Texas avec au moins 5 cartes à pique et l'enchère de 2\NT est une proposition de manche

La nomination d'une mineure est canapé et dénie 5 cartes à pique.

Si le répondant hésite à jouer 3SA, il montre son singleton au niveau de 3.

\enchbox{1\T--1\C--1\NT}
{
2\T &5-7H& 5 cartes faible\\
2\K &5-7H& 5 cartes faible\\
2\C& 8H+& 5 cartes à \P\\
2\P& 5-7H & 5 cartes à \P\\
2\NT & 8H & Naturel \\
3\T & 9H+ & Singleton \T\\
3\K & 9H+ & Singleton \K\\
3\C & 9H+ & 3 cartes à \C\\
3\P & 16H+ & 6 cartes à \P \\
3\NT && Conclusion\\
}



\end{multicols}


\titre{1\T--1\C--2\NT}

Sur toute redemande de l'ouvreur à 2\NT, 3\T est un check-back Stayman à la recherche d'un soutien de 3 cartes à \P. L'enchère de 3\P montre 6 cartes et des ambitions.

Les autres enchères au niveau de 3 montrent un singleton (3\C dans la mejeur de l'ouvreur pour le singleton \T)





\titre{
  1\T -- 1\P}

La réponse de 1\P est un signal d'alarme. La main peut être très faible. Et même si elle est forte, il ne faudra pas en attendre plus qu'un tas de points H.
L'ouvreur considère dans un premier temps que l'enchère est faible et avisera le cas échéant.

L'enchère de 2\P montre l'équivalent d'un deux fort indéterminé.

Les sauts sont forcing de manche à la recherche du meilleur contrat, jouable en face d'une main nulle.




\enchbox{1\T -- 1\P}
{
 1\NT & 12H-19H & Régulier\\
 2\T & 12-21H & Canapé\\
 2\K & 12-21H & Canapé\\
 2\C & 15-21H & 5 cartes à \C, irrégulier \\
 2\P & 22-23H & Tout type de main\\
 -> 2\NT relai obligatoire
 ->-> & 3\T & Naturel non forcing \\
 ->-> & 3\K & Naturel non forcing \\
 \rb-> & 3\C & Naturel non forcing \\
 2\NT & 22-23H & Régulier\\
 3\T & & Naturel Forcing de manche\\
 3\K & & Naturel Forcing de manche\\
 3\C & & 6 cartes à \C Forcing de manche\\
 3\P & & Bicolore 6-5 Forcing de manche\\
}

%%%%%%%%%%%%%%%%%%%%%%%%%%%%%%%%%%%%%%%%%%%%%%%%%%%%%%%%%
\titre{1\T--1\P--1\NT}

\enchbox{1\T--1\P--1\NT}
{
2\T & 0-4H & (5)6 cartes \\
2\K & 0-4H & (5)6 cartes \\
2\C & 0-7DH & 4 cartes à \C \\
2\P & 0-4H & 5+ cartes à \P \\
2\NT & 14H+ & Quantitatif\\
-> & 3\T & 12-13H\\
-> & 3\K & 14-15H\\
-> & 3\C & 16-17H\\
-> & 3\P & 18-19H\\
}

%%%%%%%%%%%%%%%%%%%%%%%%%%%%%%%%%%%%%%%%%%%%%%%%%%%%%%%%%
\titre{1\T--1\P--2\T}

Avec un jeu faible, le répondant passe ou rectifie à 2\C. Toute autre enchère est forcing de manche.

Le répondant a deux problèmes à résoudre. Connaître la force de l'ouvreur et connaître son singleton.

Le cas le plus courant, 12-13H avec les arrêts \K et \P, permet de conclure à 3\NT.





%%%%%%%%%%%%%%%%%%%%%%%%%%%%%%%%%%%%%%%%%%%%%%%%%%%%%%%%%%%%%%%%

\titre{1\T--1\P--2\C}

Cette séquence comporte un petit piège.
Avec 4 cartes à \C un jeu de 6-8DH, le répondant doit faire un petit effort à 3\C où, de toute façon il est en sécurité distributionnelle.
L'ouvreur ne doit pas croire à une main de chelem. La réponse de 1\P ayant exclu cette ambition. Si chelem, il y a, l'initiative viendra de l'ouvreur.

Avec un jeu fort (12H) et plat, le répondant peut utiliser l'enchère impossible de 2\P ou l'enchère naturelle de 2\NT. Il n'a que l'embarras du choix.

\titre{1\T--1\P--2\NT}
Sur toute redemande de l'ouvreur à 2\NT, 3\T est un check-back Stayman. On utilise les réponses standards. 

\enchbox{1\T--1\P--2\NT--3\T}
{
3\K && 5 cartes à \C et 3 cartes à \P\\
3\C && 5 cartes à \C et 3 cartes à \P\\
3\P && 4 cartes à \C et 2 cartes à \P\\
3\NT&& 4 cartes à \C et 2 cartes à \P\\
}

Les autres enchères au niveau de 3 montrent un singleton (3\C dans la mejeur de l'ouvreur pour le singleton \T)


\titre{
  1\T -- 1SA}


L'enchère de 1SA est forcing de manche avec une vraie couleur \T. Par ailleurs, l'enchère dénie 4 cartes à \C et 4 cartes à \P.

Sans rien de particulier à dire, l'ouvreur se contente de rectifier à 2\T pour voir.
Avec 5 cartes à \C, il les nomme en disant 2\C. Les enchères de 2\K et 2\P sont naturelles et promettent 5 cartes.

\enchbox{1\T--1\NT}
{
2\T &12-14H & Fitté ou pas\\
2\K &12-14H & 5 cartes \\
2\C &15H+& 5 cartes \\
2\P &12H+& Bicolore 6-5 \\
2\NT &15H+& 4 cartes à \C\\
3\T  &15H+& Beau fit \\
3\K & 15H+& naturel \\
}

Sur le redemande à 3\T, le répondant nomme son singleton, y compris à \C. Sans singleton, il propose 3\NT. Avec un jeu exceptionnel sans singleton, il a le choix entre l'enchère de 4\T qui démarre directement les négociations de chelem et l'enchère quantitative de 4\NT.

Sur la redemande à 3\K, les enchères de 3\C et de 3\P sont aussi des singletons, le fit \K étant implicite. Non fitté, le répondant propose 3\NT (ou 4\NT quantitatif) ou insiste dans sa couleur.

%%%%%%%%%%%%%%%%%%%%%%%%%%%%%%%%%%%%%%%%%%%%%%%%%%%%%%%%%%%%%%%%

\titre{1\T--2\NT}

En annonçant 2\NT, le répondant annonce une main banale de manche sans ambition de chelems. le plus souvent, l'ouvreur conclu à 4\C.

Cependant, il n'a pas limité sa main. Toute autre enchère que 4\C est une tentative de chelem. L'enchère la plus forte est 3\C qui déclenche immédiatement la procédure de contrôles. Les enchères de 3\T et 3\K sont naturelles et demande au partenaire de juger sa main. Les enchères de
3\P, 4\T et de 4\K sont des splinters, pour que le partenaire juge sa main. \textit{Avec un bicolore majeur 6-5, rien de spécial n'est prévu, hors peut-être, le Blackwood d'exclusion à 5\T et 5\K pour les experts.}

\chapter{L'ouverture de 1\K}

\begin{multicols}{2}


L'ouverture de 1\K promet 4 cartes à pique.

Elle ne peut pas comporter exactement 4 cartes à \C (on ouvrirait de 1\T).

Dans la zone 12-14H, la main peut avoir des cœurs longs qui seront annoncés en Canapé au tour suivant.

La main peut posséder cinq cartes à pique soit avec une main 5-3-3-2 de 15-17H, soit toute distribution à partir de 18H.

Le répondant dispose de deux relais, le relai à 1\C avec des mains de 5 à 11H, toute distributions et le relai à 1\P, soit avec des mains très faibles (moins de 5H) ou faible (jusqu'à 7DH) et fittées à \P, soit avec des mains plates non fittées et forcing de manche.

Les enchères de 1\NT, 2\T et 2\K sont des Texas forcing de manche. l'enchère de 2\C est Texas aussi, elle indique un fir limite de manche.


\enchbox{Ouverture de 1\K}
{
1\C & 5-11H & Pas 4 cartes à \P \\
1\P & 0-4H & Toute distribution \\
\rb & 5-7DH & 4 cartes à \P \\
& 12H+ & Pas 4 cartes à \P \\
1SA & 12H+ & 5 cartes à \T \\
2\T & 12H+ & 5 cartes à \K \\
2\K & 12H+ & 5 cartes à \C \\
2\C & 11DH+ & 4 cartes à \P \\
2\P & 8-10DH & 4 cartes à \P \\
2\NT & 12-15H & 4(5) cartes à \P sans singleton\\
3\T & 9-11H & 6+ cartes à \T \\
3\K & 9-11H & 6 cartes à \K \\
3\C & 9-11H & 6+ cartes à \C \\
3\P & 7-9H & 5 cartes à \P \\
3SA & 8-11H & Chicane indéterminée \\
4\T & 8-11H & Singleton \T\\
4\K & 8-11H & Singleton \K\\
4\C & 8-11H & Singleton \C\\
4\P & 0-10H& Barrage 6 cartes \\
4\NT && Blackwood brutal\\
}

\end{multicols}

\titre{1\K--1\C}

Le répondant a au plus une main limite de manche. De plus il n'a pas 4 cartes à \P.

Le plus souvent, on va jouer 1\NT.
La procédure est la même que sur l'ouverture de 1\T. La redemande de l'ouvreur à 1\NT indique une main régulière de 15 à 17H. Avec un jeu régulier de 12-14H, il utilise l'enchère Texas de 1\P de sorte que la main sera joué du même coté que le champ.

Les enchères sans saut sont naturelles et non forcing. A défaut d'avoir trouvé un fit à pique, on cherche un fit dans la vrai couleur.

Mise à part 2\NT, les enchères à saut sont naturelles et forcing de manche. Toutes les mains fortes non forcing de manche passe par l'enchère de 1\P. Dans la plupart des cas, le répondant annoncera 1\NT et l'ouvreur poursuivra de façon naturelle non forcing.

Ce chapitre est moins détaillé que le chapitre sur l'ouverture de 1\T car, en réalité, tout ce qui est nécessaire pour développer l'ouverture de 1\K n'est qu'un sous-ensemble de ce qui est nécessaire pour développer l'ouverture de 1\T. Se pose quand même le problème des fit \C quand le répondant est dans la zone 5-11H.

Avec 5 cartes à \P, l'ouvreur doit annoncer 1\NT dans la zone 15-17H. Avec un jeu plat de 18H, il peut répondre 2\NT. Sinon, la réponse de 1\K étant positiven, il impose la manche par l'enchère de 2\P.

\enchbox{1\K--1\C}
{
1\P & 12-14H & Régulier\\
\rb& 22H+ & Régulier\\
& 15H+ & Toute distribution \\
1\NT & 15-17H & Régulier \\
2\T & 12-14H & 5 cartes à \T \\
2\K & 12-14H & 5 cartes à \K \\
2\C & 12-14H & 5 cartes à \C \\
2\P & &5+ cartes, forcing de manche \\
2\NT & 18-19H & Régulier \\
3\T  && Canapé forcing de manche \\
3\T  && Canapé forcing de manche \\
3\C && 5-5 forcing de manche \\

}

\titre{
  1\K \\
  Réponses après intervention
}

Les Texas mineurs perdent leurs caractères forcing de manche et commencent à 8H.

\section*{Intervention par contre}

Le \Redouble est une enchère punitive. Il dénie 4 cartes à \C et promet 10H et deux couleurs 4\ieme. Tous les contre subséquents sont punitifs.

Avec un fit \P, on donne le fit à 2\P ou on fait un Texas, le tout en ignorant de \Double. L'intervention donne une troisième façon de donner le fit, l'enchère de 1\P avec un jeu de manche plat sans ambition de chelem. Attention, cette façon de faire fait jouer le coup de l'autre main.

\section*{Intervention par 1\C}

Le \Double et l'enchère de 1\P montrent a peu près le même type de jeu. Une main non fittée, avec ou sans l'arrêt  \C (dans tous les cas, on préfère faire jouer l'ouvreur). Le \double est plutôt compétitif, en tout cas limité à 11H alors que l'enchère de 1\P est forcing de manche. Par ailleurs, l'enchère de 1\P peut masquer une main fittée de manche dans un jeu plat.

\titre{1\K--1\C--1\P}
Dans cette séquence, le répondant considère, dans un premier temps que l'ouvreur possède 12-14H. Retrouver les fit \C est important.

Dans cette séquence, le répondant n'a pas le droit de dépasser 2\NT pour laisser à l'ouvreur la possibilité de développer une main forte.

\enchbox{1\K--1\C -- 1\P}
{
1\NT & 5-10H & Régulier\\
2\T & 5-10H & (5)6 cartes \\
2\K & 5-10H & (5)6 cartes \\
2\C & 5-10H & 5+ cartes à \C \\
2\P & 10-11H & 5 cartes à \C \\
2\NT & 11H & Régulier\\
}

\titre{1\K--1\C--1\NT}

\enchbox{1\K--1\C -- 1\NT}
{
2\T & 5-10H & (5)6 cartes \\
2\K & 5-10H & (5)6 cartes \\
2\C & 5-7H & 5+ cartes à \C \\
2\P & 8-11H & 5 cartes à \C \\
2\NT & 8H & Régulier\\
3\T & 9-11H & Singleton \T \\
3\K & 9-11H & Singleton \K \\
3\C & 9-11H & Singleton \C \\
3\P & 9-11H & 3 cartes à \P\\
}

\titre{1\K--1\C--2\NT}
\enchbox{1\K--1\C -- 2\NT}
{

3\T &  & Check-back \\
-> & 3\K & 3 cartes à \C et 5 cartes à \P\\
-> & 3\C & 3 cartes à \C et 5 cartes à \P\\
-> & 3\P & 2 cartes à \C et 5 cartes à \P\\
-> & 3\NT & 2 cartes à \C et 4 cartes à \P\\
3\K &  & Singleton \K \\
3\C &  & Singleton \C \\
3\P &  & Singleton \T\\
}

\titre{1\K--1\P--1\NT}

\enchbox{1\K--1\P -- 1\NT}
{
2\T & 0-4H & (5)6 cartes \\
2\K & 0-4H & (5)6 cartes \\
2\C & 0-4H & 5+ cartes à \C \\
2\P & 0-7DH & 4 cartes à \P \\
2\NT & 14H+ & Quantitatif\\
-> & 3\T & 12-13H\\
-> & 3\K & 14-15H\\
-> & 3\C & 16-17H\\
-> & 3\P & 18-19H\\
}

\titre{1\K--1\P--2\NT}
\enchbox{1\K--1\P -- 2\NT}
{

3\T &  & Check-back \\
-> & 3\K & 3 cartes à \C et 5 cartes à \P\\
-> & 3\C & 3 cartes à \C et 5 cartes à \P\\
-> & 3\P & 2 cartes à \C et 5 cartes à \P\\
-> & 3\NT & 2 cartes à \C et 4 cartes à \P\\
3\K &  & Singleton \K \\
3\C &  & Singleton \C \\
3\P &  & Singleton \T\\
}

\chapter{ L'ouverture de 1\C}

L'ouverture de 1\C indique une main d'au moins 15H sans majeure quatrième.


Expliquons la réponse de 1\P. C'est une réponse qui dénie 5 cartes à \P. C'est important pour la suite des enchères. Après cette enchère de 1\P, toute enchère à pique devient artificielle et permet d'annoncer toutes les mains qui ne rentrent pas dans les cases. On répond 1\P avec toutes les mains faibles de 0 à 5 H (sans 5 cartes à \P bien sur). Et répond aussi 1\P à partir de 9H ce qui dans le contexte est forcing de manche. Avec 5 cartes à \C, dans cette zone faible, on répond aussi 1\P.

La logique de la réponse de 1\P est donc a peu près la même que sur ouverture de 1\T ou de 1\K, à ceci près qu'elle dénie 5 cartes à pique.

Avec 5 cartes à \P et un jeu faible, il répond 2\T non forcing. A partir de 6H, il répond 2\K, forcing. il annoncera les \C éventuels plus tard si nécessaire.



\enchbox{Ouverture de 1\C}
{
1\P&0-5H ou 9H+& Pas 5 cartes à \P \\
1SA& 6-8H & Pas de majeure cinquième\\
2\T & 0-5H & 5 cartes à \P\\
2\K & 6H+ & 5 cartes à \C\\
2\C & 6H+& 5 cartes à \P\\
2\P & 3-8H & 6 belles cartes à \P\\
2\NT & 9H+ & 6 très belles cartes à \T\\
3\T & 9H+ & 6 très belles cartes à \K\\
}




\titre{1\C--1\P}

L'ouvreur fait une enchère en face d'une main potentiellement ultra-faible.
Si le répondant est fort, il reparlera et l'ouvreur avisera.

Comme dans tous le système, les enchères à saut de l'ouvreur (sauf 2\NT\ 22-23H) sont forcing de manche.
Avec une vingtaine de points seulement, l'ouvreur utilise artificiellement les enchères de 2\C et 2\P.

Les valeurs en point H sont indicative. Il faut plutôt compter en levées de jeu.



\enchbox{1\C--1\P}
{
1\NT & 15-19H & Jeu régulier \\
2\T  & 15-19H & Naturel \\
2\K  & 15-19H & Naturel \\
2\C  & 19H+ & Bicolore mineur\\
2\P  & 19-22H & Unicolore mineur\\
2\NT & 22-23H & Régulier\\
3\T &22H+& Naturel forcing de manche\\
3\K &22H+& Naturel forcing de manche\\
3\NT && Pour les jouer \\
}


\titre{1\C--1\P--1\NT}

Comme dans toutes les séquences 1x--1\P--1\NT, les enchères au niveau de 2 indiquent une misère, l'enchère de 2\NT est quantitative et les enchères au niveau de 3 montrent un singleton.

\enchbox{1\C--1\P--1\NT}
{2\T && CONCLUSION\\
2\K && CONCLUSION\\
2\C && CONCLUSION \\
2\P &9H+& Bicolore mineur\\
2SA &14H+& Quantitatif \\
3\T && Singleton \T \\
3\K && Singleton \K \\
3\C && Singleton \C \\
3\P && Singleton \P \\
3\NT&9-13H& CONCLUSION }

\enchbox{1\C--1\P--1\NT -- 2\NT}{
3\T &15H& \\
3\K &16H& \\
2\C &17H& \\
3\P &18H& \\
3\NT &19H& \\
}

Ainsi, le répondant peut soit conclure, soit rechercher un chelem, éventuellement en mineure.

\titre{1\C--2\K}

Le répondant montre 5 cartes à \C dans un jeu pas complètement nul. Avec une main banale et 2 ou 3 cartes à \C, jusqu'à 17H, l'ouvreur se contente de rectifier à 2\C. Tous les sauts sont forcing de manche. Avec un jeu fort sans enchère naturelle, on utilise le Joker à 2\P.



\enchbox{1\C--2\K}
{
2\C & 15-17H & 2 ou 3 cartes à \C\\
2\P & 18H+   & Non fitté, forcing manche \\
2\NT & 18-19H & Régulier \\
3\T  & 15-17H & Naturel \\
3\K  & 15-17H & Naturel \\
3\C  & 18H+ & Fit forcing de manche \\
}


\chapter{ L'ouverture de 1\P}

Pour les habitués de la majeure cinquième, l'ouverture de 1\P proposée ici est fort différente.

La première différence est la limitation à 17H. La seconde différence est l'exclusion des mains 5-3-3-2 du système.
Le système de réponses proposé ici obéit au cahier des charges suivants : rester le plus naturel et le plus simple possible, rester le plus cohérent avec le reste du système.

En conséquence, comme pour les autres ouvertures au niveau de 1, les changements de couleur à saut à 3\T, 3\K et 3\C sont des propositions naturelles non forcing et sur les enchères Texas de 2\T, 2\K et 2\C, l'ouvreur rectifie le Texas avec une main banale ou dévoile ses batteries avec une main forte.

Le répondant, avec une main de 0 à 7H, doit en premier lieu songer à passer.  Il faut passer, particulièrement avec deux ou trois atouts. Bien sur, les adversaires vont réveiller. Mais les réveils après une ouverture de 1\P sont inconfortables. Les laisser se dépêtrer peut conduire à de bonnes surprises.

Avec 4 atouts, il est moins astucieux de passer mais on n'a pas vraiment le choix sauf à faire un barrage à 4\P en fonction de la vulnérabilité.

Les seules mains faibles où il est habile de continuer le dialogue sont les mains avec un singleton pique. Dans ce cas, le relai à 1\NT a de bonnes chances d'améliorer la partielle.

A partir de 8H ou 9DH, le répondant peut montrer son soutien à pique de 4 manières. Avec 5 atouts, il peut faire un splinter ou un barrage. Avec 4 atouts, il peut faire un mixted raise à 3\P. S'il est certain de jouer la manche sans ambition de chelem, il donne le soutien par l'intermédiaire de l'enchère de 2\NT et l'ouvreur conclue à 4\P


Dans tous les autres cas, il donne le soutien en Texas à 2\C. Cette façon de faire en Texas laisse suffisamment d'espace pour développer toutes les mains limite de manche ou de chelem.
La seule chose importante à retenir : \textit{Ne jamais faire de soutien différé} ! Toute autre enchère que 2\C dénie un fit à pique.


Avec 5 cartes à \C (et 8H, cela va sans dire), le répondant fera un Texas à 2\K. Ici encore, la main n'est pas zonée mais la souplesse du Texas permet de poursuivre le dialogue sans difficultés.

A partir de 13H, sans 3 cartes à pique et sans 4 cartes à  à \C, on fera un relai à 2\T. Utiliser l'enchère de 2\T comme relai poubelle forcing de manche n'est pas une idée nouvelle !
Le répondant peut être unicolore \T, unicolore \K, ou même bicolore mineur.
Sur ce relai à 2\T, la redemande poubelle est 2\K (et non pas 2\P) comme si 2\T était un Texas. C'est un puppet.
Cette façon de faire est économique et cohérente, ainsi on répond de la même façon aux enchères de 2\T, 2\K et 2\C.

Il reste donc 3 enchères pour les mains de 8 à 12H sans 5 cartes à \C et sans 3 cartes à \P.

Avec 8-9H, exactement 2 cartes à \P, et sans 4 cartes à \C, on répond 2\P. C'est une préférence anticipée. Avec ces mains, on aurait rectifié à 2\P. Autant le faire tout de suite. De temps en temps, on  ratera un fit neuvième en mineur au profit d'un fit septième en majeure. C'est le prix à payer. Et lorsque les adversaires gagnent 3\C, c’est tout bénéfice. Trouver notre fit mineur les aurait aider à trouver le leur.

L'enchère la plus souple est la réponse à 1\NT. Elle peut être faible, pas nulle quand même, avec un singleton \P. Avec 4 cartes à \C, elle commence à 8H et n'a pas de limite supérieure. Avec deux cartes à \P et sans 4 cartes à \C, elle va de 10 à 12H.


\enchbox{1\P}
{
1\NT & 5H+ & Dénie 3 cartes à \P. Forcing, illimité si 4 cartes à \C \\
2\T  & 12H+ & Forcing de manche, pas de fit, pas 4 cartes à \C \ \\
2\K  & 8H+  & 5 cartes à \C \\
2\C  & 9DH+ & 3+ cartes à \P\\
2\P  & 8-9H & 2 cartes à \P, préférence anticipée  \\
2\NT & 12-15H & Pour jouer 4\P \\
3\T  & 9-11H & 6 belles cartes \\
3\K & 9-11H & 6 belles cartes \\
3\C & 9-11H & 6 belles cartes \\
3\P & 9-10DH & 4 cartes \\
3SA &7-11H& Chicane indéterminée et 5 atouts \\
4\T &7-11H& Splinter 5 atouts \\
4\K &7-11H& Splinter 5 atouts \\
4\C &7-11H& Splinter 5 atouts \\
4\P &0-10H& Barrage 5 atouts \\
}




\titre{1\P--2\P}

Danger : l'enchère promet deux cartes et non pas trois.

Le répondant possède 8-9H. L'ouvreur passe le plus souvent. Il peut conclure à 4\P avec 6 cartes ou à 3\NT avec 16-17H.

Il peut montrer sa deuxième couleur à la recherche d'un arrêt dans son singleton s'il veut jouer 3\NT. Mais il faut une grosse distribution pour procéder ainsi. En points d'honneur, la manche mineure est vraiment lointaine.

S'il hésite, il peut toujours proposer 2\NT ou 3\P.


\titre{1\P--2\C}

\begin{multicols}{2}



La réponse de 2\C annonce un fit, à partir de 8H mais peut provenir d'une main très forte. Le répondant a besoin de connaître la force de l'ouverture soit en vue de la manche, soit pour jouer un chelem.
L'ouvreur n'a que deux réponses possibles et c'est le répondant qui poursuit le dialogue en fonction de ses ambitions. La réponse de 2\NT est de facto forcing de manche. L'ouvreur se méfiera des mains de 15H avec un honneur sec.

\enchbox{1P--2\C}
{2\P & 11-15H& \\
2\NT & 15-17H &\\}



Les redemandes du répondant au niveau de 3 sont naturelles et demandent un complément pour jouer le chelem. 3\NT est une demande de contrôle brutale. Les splinters différés sont très fort et demandent à l'ouvreur d'aller au chelem sauf points perdus dans la couleur.

Il faut bien comprendre que la séquences 1\P--2\C--2\P--3\T et 1\P--2\C--2\NT--3\T remplacent les soutiens différés des systèmes classiques 1\P--2\T--2x--3\P. Toutefois, la séquence est à la fois plus naturelle (les trèfles sont toujours réels), plus économique et plus précise.

La redemande à 2\NT est quantitative de manche et la redemande à 3\P est quantitative de chelem.

Sur la proposition de manche à 2\NT, l'ouvreur peut revenir à 3\P, nommer la manche ou, s'il hésite, dire 3\T qui ne veut rien dire d'autre que : «Partenaire, nomme la manche avec 12 (ou 10) et revient à 3\P avec 11 (ou 9)». Dans ce contexte, inutile de raconter sa vie.



\enchbox{1\P--2\C--2\P}
{
\Pass & 9-10 DH&\\
2\NT & 11-12 DH&\\
3\T  && 5 cartes, chelem \\
3\K  && 5 cartes, chelem \\
3\C  && 5 cartes, chelem \\
3\P  &19-20DH& Quantitatif \\
3\NT &21DH+& Contrôles \\
4\T  && Splinter fort \\
4\K  && Splinter fort \\
4\C  && Splinter fort \\
4\P & 13-20 DH &\\
}

\enchbox{1\P--2\C--2\NT}
{\
3\T  && 5 cartes, chelem \\
3\K  && 5 cartes, chelem \\
3\C  && 5 cartes, chelem \\
3\P  &15-16DH& Quantitatif \\
3\NT && Contrôles \\
4\T  && Splinter fort \\
4\K  && Splinter fort \\
4\C  && Splinter fort \\
4\P & 9-16 DH &\\
}


\end{multicols}
\titre{1\P--3\P}

Le soutien n'est pas un pur barrage, c'est un mixted raise, 4 atouts et 9-10 DH. Ces mixted raise ont remplacé petit à petit les barrages faibles. C'est plus efficace et plus précis.

A priori pas de chelem en vue mais, si besoin était, du genre bicolore 6-6, avec des adversaires endormis l'ouvreur peut toujours déclencher les contrôles à 3\NT ou annoncer sa deuxième couleur.


\titre {1\P -- 1\NT}

Le relai à 1\NT est forcing un tour et invite l'ouvreur à décrire son jeu. Cette description se fait de façon économique sauf avec une main distribuée dans la zone 16-17H.

Attention, l'ouvreur est limité à 17H. Donc avec les mains de 0 à 7H, il faut en priorité songer à passer.  Avec un singleton ou une chicane pique et un jeu pas trop nul (avec une main nulle, les adversaires vont faire au moins un tentative de manche), le relai à 1\NT permettra le plus souvent de trouver une meilleure partielle.

Le plus souvent, l'enchère de 1\NT provient d'un jeu de 8 à 12H sans 4 cartes à pique.

Toutefois, l'enchère de 1\NT peut aussi provenir d'une main avec exactement 4 cartes à \C (ou 5 cartes moches dans un jeu faible). Dans ce cas, il n'a pas de limite supérieure.

\enchbox{1\P--1\NT}
{
2\T & 11-17H & 4+ cartes à \T \\
2\K & 11-17H & 4+ cartes à \K \\
2\C & 11-17H & 4+ cartes à \C \\
2\P & 11-17H & 6 cartes à \P \\
2\NT & 16-17H & 6 cartes à \P et une mineure 4\ieme \\
3\T & 16-17H & 5 cartes à \T \\
3\K & 16-17H & 5 cartes à \K \\
3\C & 16-17H & 5 cartes à \C \\
3\P & 16-17H & 7 cartes à \P ou 6 très belles\\
}


\titre{1\P--1\NT--2\T}

L'ouvreur possède 5 cartes à pique et 4 cartes à \T.

Si le répondant a un jeu faible, il passe ou redemande 2\K dans sa longueur.

Avec 8-10H, il fait une enchère de préférence, 2\P avec deux cartes à \P ou 3\T avec 4 cartes à \T et singleton \P.

Avec 11-12H, le répondant redemande 2\NT.

Avec 13H+, s'il ne souhaite pas conclure à 3\NT le répondant commence par dire 2\C montrant 4 cartes dans une belle main. Une enchère subséquente à \K de sa part doit être comprise comme naturelle, Canapé, avec des intentions de chelem. De même, une enchère à \T indiquera un fit forcing de manche.

Le plus souvent, c'est l'arrêt carreau qui manque pour jouer à \NT. De temps à autre, il y a un chelem à jouer.



Cette séquence 1\P--1\NT--2\K est similaire. La différence, c'est qu'une longueur à \T dans un jeu faible sera exprimée au niveau de 3.

\titre{1\P--1\NT--2\C}

Cette séquence est une plaie dans tous les systèmes. La main de l'ouvreur va jusque 17H. Le répondant se sent dans l'obligation de parler à partir de 8H. Mais l'ouvreur, ne sachant pas s'il en a 8 ou 10 doit tenter le coup à pile ou face. Avec le deux sur un forcing de manche, la situation est encore pire puisque la main du répondant va jusque 11H.

En trèfle rouge, la main du répondant est illimité mais \dots\ on a une solution. L'enchère de 2\P dans cette séquence est artificielle ! Elle promet un fit \C d'au moins 10H.


\enchbox{1\P--1\NT--2\C}{
Passe & 0-7H & 3+ cartes à \C \\
2\P & 10H+ &  4 cartes à \C\\
2SA & 10-12H & Régulier\\
3\T & 7-8H & Naturel \\
3\K & 7-8H & Naturel \\
3\C & 8-9H & 4 cartes à \C

}

\enchbox{1\P--1\NT--2\C--2\P}
{
2\NT & 15-17H& Belle main \\
3\C &11-12H& \\
4\C & 13-14H& \\
}


\titre{1\P--2\T}

La réponse à 2\T impose la manche. La possibilité d'un fit \P existe si l'ouvreur possède 6 cartes. La possibilité d'un fit \C existe si l'ouvreur possède 5 cartes.
En aucune façon, l'ouvreur ne nomme des cœurs 4\ieme.

\enchbox{1\P -- 2\T}
{
2\K &11-17H & RAS \\
2\C &11-17H& 5+ cartes à \C \\
2\P &11-17H& 6+ cartes à \P \\
2\NT & 16-17H & 6 cartes à \P et une mineure 4\ieme \\
3\T & 16-17H & 5 cartes à \T \\
3\K & 16-17H & 5 cartes à \K \\
3\C & 16-17H & 5 cartes à \C \\
3\P & 16-17H & 7 cartes à \P ou 6 très belles\\
}

La deuxième enchère du répondant est naturelle. Toutefois, il faut comprendre que le répondant a dénié 4 cartes à \C. Toute enchère à \C de sa part doit être considérée comme artificielle sans enchère naturelle. L'enchère remplace une 4\ieme forcing.

\chapter{L'ouverture de 1\NT}

L'ouverture de 1NT provient d'une main de 12-14H sans majeure quatrième mais éventuellement avec une majeure cinquième. Beaucoup de mains 5-4-2-2 ou 6-3-2-2 sont éligibles à l'ouverture de 1\NT. Notamment, les mains de 12-13H avec 5 cartes à cœur, une mineure quatrième et deux doubletons gardés. A partir de 14H, on utilisera plutôt une ouverture au niveau de 2 avec ce genre de main semi-régulière.

Avec un fit 5-3 en majeure, une main régulière en face d'une main régulière, il n'est pas prioritaire de jouer en majeure. Par contre, rater un fit 4-4 en majeure das cette zone est à la fois grave et hors champ. En tournoi par paires, c'est un zéro garanti. C'est sans doute le plus grand frein à l'adoption du \NT faible.  Pour contourner cet écueil, on n'ouvre jamais de 1\NT avec une majeure quatrième. Cela diminue fortement la fréquence d'utilisation de l'enchère. Mais comme la zone 12-14 est beaucoup plus fréquente que la zone 15-17, on est à l'équilibre.

Évidemment, il n'y a pas de Stayman ! La  réponse de 2\T est une demande de majeure cinquième. On peut la qualifier de \textit{Puppet Stayman}. On l'utilisera avec presque toutes les mains limites de manche. Reparler après un Texas est presque toujours forcing de manche et l'enchère de 2\NT, bicolore mineur est forcing de manche.
Avec une courte à \T et un jeu faible ou moyen, il est possible d'utiliser 2\T avec l'intention de passer sur toute réponse.

Avec un bicolore majeur 5-5, le répondant utilise la séquence 1SA-2\K-2\C-2\P avec un jeu limite de manche. C'est la seule exception à la règle, toute enchère est forcing de manche après un Texas. Avec un jeu de manche, il passe par la réponse directe à 4\K. Et avec un jeu de chelem, il utilise la séquence 1SA-2\C-2\P-3C.

\enchbox{Ouverture de 1\NT}
{
2\T &0H+& Puppet Stayman \\
-> &  2\K & Pas de majeure cinquième\\
\rw -> & 2\C & 5 cartes à \C \\
 -> &  2\P & 5 cartes à \P\\
2\K &0H+& Texas \C \\
2\C &0H+& Texas \P \\
2\P &0H+& Texas \T\\
2\NT &12H+& Bicolore mineur\\
3\T &0H+& Texas \K \\
3\K &16+& Texas \C de chelem\\
3\C &16H+& Texas \P de chelem\\
3\NT &12H+& Pragmatique\\
4\K & 12H+& 5-5 majeur de manche \\
}



%%%%%%%%%%%%%%%%%%%%%%%%%%%%%%%%%%%%%%%%%%%%%%%%%%%%%%%%%%%%%%%%%%%
\titre{1\NT -- 2\T -- 2\K}

L'ouvreur n'a pas de majeure cinquième. Le répondant va maintenant expliquer ses intentions. Avec un singleton, il l'annonce. Si l'ouvreur ne peut pas conclure à 3SA, il sera toujours temps de trouver un contrat de substitution.

\enchbox{1\NT -- 2\T -- 2\K}
{
2\C & 10-11H & 5 cartes à \C \\
2\P & 10-11H & 5 cartes à \P \\
2\NT & 11HL & Proposition \\
3\T  & 12H+ & Singleton \T\\
3\K & 12H+ & Singleton \K \\
3\C & 12H+ & Singleton \C \\
3\P & 12H+ & Singleton \P \\
}


\newpage

\titre{1\NT -- 2\K -- 2\C}

Le répondant possède au moins 5 cartes à \C. Il n'a pas 5 cartes à \P. S'il possède 6 cartes à \C, il n'a pas d'ambitions de chelem.


En dehors des conclusions à 3SA et 4\C, le répondant a deux façon de procéder. Il peut annoncer un bicolore au moins 5-5. Les enchères naturelles de 3\T et 3\K sont utiles à cet effet.

Avec un singleton dans une main 5-4-3-1 (ou plus rarement 5-4-4-0), le répondant annonce son singleton et l'ouvreur avisera. Pour annoncer le singleton, le répondant commence par dire 2\NT, l'ouvreur fait un relai à 3\T et le répondant annonce la couleur de son singleton (3\C = singleton \T).


\enchbox{1\NT -- 2\K -- 2\C}{
2\P && 5-5 limite de manche\\
-> & 2\NT & relai\\
->->&& 3\T\ 5 cartes limite  \\
->->&& 3\K\ 5 cartes limite  \\
->->&& 3\C\ 5-5 majeur limite  \\
2\NT && J'ai un singleton \\
-> & 3\T\ ->& relai\\
->->&& 3\K singleton \K\\
->->&& 3\C singleton \T\\
->->&& 3\P singleton \P\\
3\T && 5 cartes\\
3\K && 5 cartes\\
3\C && 6 cartes limite de manche\\
3\NT && Choix de manche\\
4\C && Fin\\
}


\newpage

\titre{1\NT -- 2\C -- 2\P}

Le répondant possède au moins 5 cartes à \P. S'il possède 6 cartes à \P, il n'a pas d'ambitions de chelem. S'il possède 5 cartes à \C, il aura des ambitions de chelem.

D'autre part, il ne possède pas une main limite de manche (2\T serait obligatoire dans ce cas). Donc toutes les enchères sont forcing de manche.

En dehors des conclusions à 3SA et 4\P, le répondant a deux façon de procéder. Il peut annoncer un bicolore au moins 5-5. Les enchères naturelles de 3\T et 3\K sont utiles à cet effet.

Avec un singleton dans une main 5-4-3-1 (ou plus rarement 5-4-4-0), le répondant annonce son singleton et l'ouvreur avisera. Pour annoncer le singleton, le répondant commence par dire 2\NT, l'ouvreur fait un relai à 3\T et le répondant annonce la couleur de son singleton (3\P = singleton \T).



\enchbox{1\NT -- 2\C -- 2\P}{
2\NT && J'ai un singleton\\
-> & 3\T\ >& 3\K singleton \K\\
\rb -> & 3\T\ >& 3\C singleton \C\\
-> & 3\T\ >& 3\P singleton \T\\
3\T && 5 cartes\\
3\K && 5 cartes\\
3\C && 5-5 de chelem\\
3\NT && Choix de manche\\
4\P && Fin\\

}

\newpage

\titre{1\NT -- 2\P}

Le répondant possède 6 cartes à \T ou 5 cartes et une volonté de chelem (19-20H).

L'ouvreur peut utiliser l'enchère de 2\NT avec deux gros honneurs troisièmes. Sinon il rectifie à 3\T. Le répondant nomme alors éventuellement son singleton.

\newpage

\titre{1\NT -- 2\NT}

L'enchère de 2\NT indique un bicolore mineur. Sans majeure cinquième, l'ouvreur peut fixer l'atout en disant 3\T ou 3\K.
Le répondant annonce alors son singleton à 3\C ou 3\P, ce qui laisse l'occasion à l’ouvreur de freiner avec des points perdus.

Avec une majeure cinquième, l'ouvreur nomme sa majeure.

\newpage

\titre{1\NT -- 3\T}


Le répondant possède 6 cartes à \K ou 5 cartes et une volonté de chelem (19-20H).

L'ouvreur rectifie et le répondant annonce son singleton. 3\NT avec un singleton \T sans volonté de chelem.

\newpage

\chapter{L'ouverture de 2\T}

\begin{multicols}{2}



L'ouverture de 2\T indique au moins 5 cartes à trèfle et entre 12 et 14H. Pas de majeure 4\ieme.
Attention, dans cette zone, les mains 5-3-3-2 s'ouvrent systématiquement de 1SA.
Les bicolores mineurs 5-4-2-2 de 12-13H s'ouvrent systématiquement de 1SA de même que les unicolores 6-3-2-2. De sorte que l'ouverture de 2\T garanti 14H ou un singleton.

Avec une majeure 5\ieme et une main au moins limite de manche, le répondant utilise un relai à 2\K.

Avec une solide majeure 6\ieme dans une main limite, le répondant propose de jouer 3 ou 4 dans sa majeure en la nommant au niveau de 3. Le contrat de 3SA est exclu. Un retour dans la mineure est exclu sauf cas d'espèce.

La logique du système est la suivante : dès qu'un fit mineur est connu au palier de 3, si l'ouvreur reparle, soit parce que la situation est forcing, soit parce qu'il est maximum dans une situation limite de manche, l'ouvreur nomme son singleton. On nomme toujours la couleur du singleton mais, il arrive que, faute de place, un des singletons soit nommé indirectement.

Cette ouverture est censé être un point fort du système. Il quelques petits détails à mémoriser mais leur oubli ne sera pas forcément dramatique.

\enchbox{Ouverture de 2\T}
{
2\K & 10H+ & Relai \\
2\C & 3-8H & 6 belles cartes à \C \\
2\P & 3-8H & 6 cartes cartes à \P\\
2SA & 11H & Proposition\\
3\T & 5-9H & Au moins 3 cartes \\
3\K & 12H+ & Naturel \\
3\C & 9-11H & Belle couleur 6\ieme\\
3\P & 9-11H & Belle couleur 6\ieme\\
3\NT && Conclusion \\
4\T && Chelem \\
}



\subsection*{Le relai à 2\K}

L'enchère de 2\K relai est une sorte de troisième couleur forcing. Toutes les mains limites de manches fittées à trèfle ou possédant une majeure 5\ieme passe par ce relai. Avec 14H, l'ouvreur sait que la manche sera atteinte d'une façon ou d'une autre.
\\
\enchbox{2\T -- 2\K}
{
2\C &12-14H & 3 cartes à \C \\
2\P &12-14H & 3 cartes à \P (pas à \C)\\
2SA &12-13H & 6-4 médiocre\\
3\T &12-14H & 7 cartes à \T \\
3\K &12-14H & Bicolore 6-5\\
3\C &14H& 6-4 singleton \C\\
3\P &14H& 6-4 singleton \P\\
3SA &14H& 2-2-4-5 ou 2-2-3-6\\
}






\subsection*{La séquence 2\T--2\K--2\C}

 La redemande de l'ouvreur à 2\C promet 3 cartes à \C. Il peut aussi avoir 3 cartes à pique.

 Si le répondant était intéressé par les piques, il les annonce. la redemande à 2\P est forcing un tour.

 L'enchère de 3\P est une convention 2012, elle fixe l'atout \P dans l'optique d'un chelem.

 Il y a deux façons d'annoncer un fit limite, 2\NT et 3\T. Cela permet d'orienter les \NT, on l'espère, de la meilleure main.
\\
 \enchbox{2\T--2\K--2\C}
 {
  2\P  && 5 cartes à \P\\
  2\NT && Fit \T limite \\
 3\T &10-11H & Fit \T limite\\
 3\K & 12H+& Fit \T forcing manche\\
 3\C &10-11H& Proposition à \C\\
 3\P && Chelem à \C\\
 }

\subsection*{La séquence 2\T--2\K--2\C--2\P}

 S'il est fitté l'ouvreur répond 3\K. Ce qui laisse la place au répondant de proposer la manche en disant 3\P ou d'explorer le chelem à 3\C.
 Non fitté, l'ouvreur utilise l'enchère de 2\NT avec 4 cartes à \K (un chelem est encore possible dans cette couleur) ou revient à 3\T dans 6 cartes.

 \textit{Facultatif : Avec 14H, l'ouvreur utilise les enchères de 3\C et de 3\P comme des échos des enchères de 2\NT et de 3\T.}

 Comme d'habitude l'enchère de 3SA montre une main 2-3-3-6 et donc de 14H.


 \enchbox{2\T--2\K--2\C--2\P}
 {
 2\NT & 12-13H& 1\P-3\C-4\K-5\T \\
 3\T  & 12-13H& 1\P-3\C-3\K-6\T \\
 3\K  & 12-14H& 3\P-3\C-1\K-6\T \\
 \rb -> & 3\C & Chelem à \P \\
 -> & 3\P & Proposition \\
 3\C & 14H & 1\P-3\C-4\K-5\T \\
 3\P & 14H& 1\P-3\C-3\K-6\T \\
 3\NT &14H & 2\P-3\C-3\K-6\T \\
 }

 \subsection*{La séquence 2\T--2\K--2\C--3\T}

 En disant 3\T, le répondant montre qu'il n'était pas intéressé par les majeures mais veut juste explorer la manche. Il n'a pas dit 2\NT, l'ouvreur doit donc se méfier. Avec 14H sans singleton, il peut conclure. Sinon, avec 14H toujours, il annonce son singleton, 3\K avec un singleton \K et 3\K avec un singleton \P.

 \subsection*{La séquence 2\T--2\K--2\C--3\K}

 Dans cette situation, l'ouvreur annonce toujours son singleton. 3\C avec un singleton \K (faute de place), 3\P avec un singleton \P et 3\NT sans singleton.

\subsection*{La séquence 2\T--2\K--2\P}

 La redemande de l'ouvreur à 2\P promet annonce 3 cartes à \P et dénie 3 cartes à \C, le répondant dispose de trois enchères propositionnelle 2\NT, 3\T et 3\P.

 On ne dispose plus de l'enchère forcing à 2\NT donc 3\K sert de fourre-tout pour les chelems nébuleux.


 \enchbox{La séquence 2\T--2\K--2\P}
 {
 2\NT &11H& Proposition \\
 3\T & 10-11H& Fit \T limite  \\
 3\K & 12H+ & Fit \T forcing \\
 3\C && Chelem à \P \\
 3\P &11H& Proposition \\
 }

  \subsection*{La séquence 2\T--2\K--2\P--3\K}

  Dans cette situation, l'ouvreur annonce toujours son singleton. 3\C avec un singleton \C, 3\P avec un singleton \K (faute de place) et 3\NT sans singleton.

\subsection*{La séquence 2\T--2\K--2\NT}

 L'ouvreur annonce son bicolore mineur. Il est toujours temps de conclure à 4\C ou 4\P avec une couleur jouable seul et des ambitions de chelem qui tombent à l'eau.

 Le répondant peut maintenant produire les enchères non forcing de 3\T ou de 3\K.

 Il peut démarrer les enchères de chelem en fixant l'atout à 4\T ou 4\K.

 Les enchères de 3\C et de 3\P sont des essais sans conviction vers 3\NT. La couleur annoncée est bien gardée et l'ouvreur avise. On peut imaginer un chelem si miracle (chicane dans l'autre majeure par exemple).

 \subsection*{La séquence 2\T--2\K--3\T}

 L'ouvreur annonce 7 cartes à \T.

 On dispose d'un gadget facultatif à 3\K pour connaître le singleton.

 \subsection*{La séquence 2\T--2\K--3\T--3\K}

 3\C et 3\P annoncent un singleton.

 3SA annonce une main 2-2-2-7 (le contrat sera bien joué de la bonne main le plus souvent)

 4\T montre une huitième carte et donc un singleton \K

 \subsection*{La séquence 2\T--2\K--3\K}

 En cas de misfit grave, le répondant peut proposer le contrat de 4\T. Toute autre enchère est forcing de manche. S'il veut jouer un chelem à carreau, le répondant fixe l'atout immédiatement en disant 4\K. Sinon, on part du principe que l'atout est trèfle.

\end{multicols}


\chapter{L'ouverture de 2\K}

\begin{multicols}{2}
L'ouverture de 2 \K indique au moins 5 cartes à carreau et entre 12 et 14H. Pas de majeure 4\ieme.
Attention, dans cette zone, les mains 5-3-3-2 s'ouvrent systématiquement de 1SA.
Les bicolores mineurs 5-4-2-2 de 12-13H s'ouvrent systématiquement de 1SA de même que les unicolores 6-3-2-2. De sorte que l'ouverure de 2\K garanti 14H ou un singleton.

Le problème des carreaux se pose ici. On perd la possibilité de s'arrêter à 2\C qui devient relai. De même après la réponse à 2\P sur le relai à 2\C, on perd l'enchère naturelle de 2SA qui devient un relai pour retrouver les \C.

A part cela, la logique de l'ouverture de 2\K, ainsi que les petites astuces, est la même que celle de l'ouverture de 2\T.

\enchbox{Ouverture de 2\K}
{
2\C  &10H+ & Relai ambiguë\\
2\P & 3-8H & 6 cartes cartes à \P\\
2SA & 11H & Proposition\\
3\T & 12H+ & Naturel\\
3\K & 5-9H & Au moins 3 cartes \\
3\C & 9-11H & Naturel, belle couleur 6\ieme\\
3\P & 9-11H & Naturel, belle couleur 6\ieme\\
}

\enchbox{2\K--2\C--}
{2\P &12-14H & 3 cartes à \P \\
2SA &12-14H & 3 cartes à \C (pas à \P)\\
3\T &12-14H &  4-5 cartes\\
3\K &12-14H &  7 cartes\\
3\C &14H& 5-5 singleton \C\\
3\P &14H& 5-5 singleton \P\\
3SA &14H& 2-2-4-5 ou 2-2-3-6 \\
}


 \subsection*{La séquence 2\K--2\C--2\P}

 Quand l'ouvreur annonce 3 cartes à \P, il peut encore avoir à cartes à \C. Pour le savoir, le répondant doit faire un relai à 2\NT.

 S'il est fitté, l'ouvreur redemande 3\K, ce qui laisse la place au répondant de proposer la manche à 3\C ou d'engager un chelem à \C en disant 3\P.

 \enchbox{2\K--2 \C--2\P}
 {
 2\NT && 5 cartes à \C \\
\rw -> & 3\T & Singleton \C \\
 \rw-> & 3\K & 3 cartes à \C \\
\rw -> & 3\NT & 3-2-6-2 (14H) \\
\rw -> & 4\K & 7+ cartes à \K \\
\rb 3\T & & Fit \K forcing \\
 \rw 3\K & & Fit \K limite\\
 \rb 3\C && Proposition \\
 \rw 3\P && Chelem à \C \\
 }

  \subsection*{La séquence 2\K--2\C--2\P--2\NT}

 3\T : non fitté, 3-1-5-4 ou 3-1-6-3 Faute de place, l'ambiguïté ne peut pas être levée. Le répondant peut faire une enchère non forcing à 3\K. Il peut même passer. Au delà, toutes les enchères sont forcing de manche.

 3\K : 3 cartes à \C, l'enchère de 3\C devient propositionnelle et celle de 3\P est une enchère de chelem.

 \subsection*{ La séquence 2\K--2\C--2\P--3\T}

 3\K = 7 cartes, 3\NT improbable

 3\C = singleton \C

 3\P = singleton \T

 3SA = 3-2-6-3 et 14H

 \subsection*{ La séquence 2\K--2\C--2\NT}

 L'ouvreur possède 3 cartes à \C.

 L'enchère de 3\T indique un fit \K forcing de manche (on cherchait les \P)

 L'enchère de 3\K est non forcing.

 L'enchère de 3\C est une proposition de manche.

 L'enchère de 3\P explore le chelem, convention 2012.

 On a perdu la possibilité de s'arrêter à 2\NT. Le contrat de 3\K reste un bon compromis en cas de misfit.

 \subsection*{ La séquence 2\K--2\C--3\T}

 L'enchère est ambiguë, le plus souvent, il s'agit d'un bicolore 6-4 mais avec un 5-5 moche, on ne peut pas imposer la manche. Il y a toujours 10 cartes en mineure.

 On peut explorer le chelem en fixant l'atout au niveau de 4.

 On peut passer ou donner une préférence à \K.

 Les enchères de 3\C et de 3\P cherchent à récupérer un contrat à \NT.

\end{multicols}

\chapter{l'ouverture de 2\C}

\begin{multicols}{2}


L'ouverture de 2\C est très précise, elle va de 12 à 14H. L'enchère promet 5 cartes à \C et dénie 4 cartes à \P. Mis à part en troisième position, il n'est pas recommandé d'ouvrir les mains de 11H. Comme toute enchère faite au niveau de deux, la performance est dans la précision.

Avec une main 5-3-3-2, on ouvre systématiquement de 1\NT. Avec une main 5-4-2-2, on ouvrira parfois de 1\NT. Dans une main de 12-13H, dès qu'il y a un honneur autre que l'As dans chaque doubleton, on ouvre de 1\NT. Avec une main de 14H, il est possible d'ouvrir de 1\NT avec une main particulièrement laide, typiquement avec RD et Rx comme doubletons par exemple.

Le plus souvent, la main sera donc 5-4-3-1 ou 6-3-2-2. Mais elle peut comporter une mineure 5\ieme ou 6\ieme et les cœurs peuvent très longs.

Toutes les réponses de 2\NT à 3\P sont naturelle et limite de manches (y compris le soutien à 3\C qui n'est pas un barrage). Les enchères de 3\NT et 4\C sont conclusives. Les réponses de 4\T et de 4\K n'existent pas. \textit{On peut convenir que ce sont des splinters}.

Un relai forcing de manche à 2\P existe dont l'utilité principale est de retrouver un fit à pique mais qui peut aussi démarrer des séquences de chelem.

\enchbox{Ouverture de 2\C}
{
2\P &12H+& Interrogative à \P \\
2\NT & 11HL & Naturel \\
3\T  & 9-11H & 6 belles cartes \\
3\K & 9-11H & 6 belles cartes \\
3\C & 11DH & propositionnel \\
3\P & 9-11H & 6 belles cartes \\
3SA && Conclusion \\
4\C && Conclusion \\
4\P && Conclusion \\
4\NT && Blackwood brutal \\
}

\end{multicols}

\titre{2\C--2\P}
\begin{multicols}{2}


Le répondant peut relayer à 2\P pour plusieurs raisons. Il peut chercher à localiser le singleton pour éviter un mauvais contrat à \NT.
Avec 5 cartes à \P, il peut aussi rechercher un fit. Plus rarement, il souhaite jouer un chelem.

L'ouvreur dévoile en premier son nombre de cartes à \P.

Sur toute autre redemande que 2\NT, le répondant connaît la distribution de l'ouvreur à une carte près. Sur la redemande à 2\NT, il reste une incertitude sur la présence ou non d'un singleton mineur ou d'une mineur 4\ieme.

Quoi qu'il en soit, la situation est forcing de manche, des enchères naturelles permettront de trouver la meilleure manche lorsque 3\NT est inacceptable.

\noindent
\enchbox{2\C--2\P}
{
2SA && 5 cartes à \C et 3 cartes à \P \\
3\T && 4 cartes à \T et 0-2 cartes à \P \\
3\K && 4 cartes à \K et 0-2 cartes à \P \\
3\C && 6 cartes à \C et 0-2 à \P   \\
3\P && 6 cartes à \C et 3 cartes à \P\\
3SA && (n'existe pas)\\
4\T && 6 cartes \\
4\K && 6 cartes \\
4\C && 8 cartes \\
}


\end{multicols}
\titre{2\C--2\P--2\NT}

\begin{multicols}{2}
L'ouvreur vient d'annoncer qu'il a 3 cartes à \P. Le plus souvent, le répondant aura 5 cartes lui-même et sera en mesure de conclure.

Les enchères de 3\NT et de 4\P sont des conclusions, de même que l'enchère bizarre de 4\C (le répondant cherchait sans doute un chelem miracle).

Comme la situation est forcing de manche, les enchères de 3\C et de 3\P fixent l'atout pour explorer le chelem.

Les enchères de 3\T et de 3\K sont les plus difficiles. Elles sont naturelles bien sur mais peuvent provenir de mains très différentes.
Le répondant peut craindre une entame dans l'autre mineure où l'ouvreur pourrait avoir un singleton. Il peut lui-même avoir un singleton pique et souhaite vérifier que les 3 cartes en face sont solides. Plus rarement, il peut avoir envie de jouer un chelem dans sa mineure.
Quoi qu'il en soit, l'ouvreur nomme 3\NT s'il garde correctement les deux couleurs restantes. Avec une seule couleur gardée, il la nomme (sur 3\K, il dit 3\C quand il garde les \T à cause du classique problème des carreaux). S'il est faible dans les deux couleurs restantes, le contrat de 3\NT est exclu, l'ouvreur se contentera de soutenir la mineure.


\end{multicols}
\titre{2\C--2\P--3\T}
\begin{multicols}{2}
Rares sont les cas où le répondant ne peut pas conclure. Il peut montrer 6 cartes à \P en situation forcing en disant 3\P mais comme cette enchère ne fixe pas l'atout, ce n'est pas une tentative de chelem. Par contre, les enchères de 3\C et 4\T fixent l'atout et déclenche l'exploration de chelem.

La redemande à 3\K est naturelle et il est temps pour l'ouvreur de se rappeler que rien n'a été promis à \P, une conclusion à 3\NT ne peut se faire qu'avec l'arrêt pique. S'il ne sait pas quoi dire, l'enchère de 3\P devient une quatrième forcing.

La séquence 2\C--2\P--3\K est similaire.



\end{multicols}
\titre{2\C--2\P--3\C}
\begin{multicols}{2}

Pas beaucoup de place pour explorer. Toutefois, le répondant a tous les éléments pour conclure. Attention, les redemandes à 4\T et 4\K ne sont pas des contrôles mais des enchères naturelles ! S'il veut jouer un chelem à \C, le répondant fera un Blackwood brutal. S'il veut jouer à \P, il commencera par les exprimer au niveau de 3.

\end{multicols}

\titre{2\C--2\P--3\P}
\begin{multicols}{2}

Dans cette séquence, les redemandes à 4\T et 4\K sont naturelles et dénient un intérêt pour les piques.

La redemande de 3\NT est naturelle.

Il n'y a plus de place pour explorer le chelem.  S'il veut jouer un chelem, le plus simple consiste à fixer l'atout au niveau de 5 par l'enchère de 5\C ou de 5\P. L'ouvreur ne devrait pas avoir du mal à juger de la valeur de sa main.

\end{multicols}

\titre{2\C--2\NT}

\begin{multicols}{2}

Avec 12-13H et une distribution raisonnable, l'ouvreur passe.
Avec 14H, l'ouvreur peut annoncer 3\P dans 3 cartes car le répondant pourrait y avoir 5 cartes. De même, il peut annoncer sa 6\ieme carte à \C
Sinon, il peut conclure à 3\NT. Dans cette zone, on n'a pas les moyens de s'enquérir des arrêts.

Les enchères de 3\T et de 3\K, encourageantes mais non forcing, montrent des distributions vraiment irrégulières disqualifiant le contrat de 3\NT.
\end{multicols}


\chapter{Ouvertures de 2SA et au delà}

\section*{Ouverture de 2\NT}

L'ouverture habituelle de 20-21H convient parfaitement.
On peut y greffer n'importe quel système. Ce qui est proposé dans le SEF 2024 est déjà suffisamment efficace. L'ouverture peut comporter une majeure cinquième mais le Puppet Stayman, dans sa version standard est fortement déconseillé, il crée plus de problèmes qu'il n'en règle.

On trouve quelques Staymans performants à droite et à gauche comme le Muppet, le Romex ou l'Advanced Stayman. Tout ces systèmes performants demandent un gros investissement en temps de mémorisation.

\section*{Ouverture de barrage}

Les ouvertures de barrage sont indépendantes de tout système. Jouez comme bon vous semble. Pour ma part, je reste convaincu, que les barrages constructifs sont plus efficaces que les barrages anarchiques, mais pas de beaucoup. Sauf Vert contre Rouge où il faut y aller franco.

Les barrages en Texas ou en double Texas sont des aberrations qui ne font qu'aider les adversaires à se défendre correctement.
Mais si ça vous amuse, c'est marrant à jouer. Et c'est le plus important.

\end{document}








