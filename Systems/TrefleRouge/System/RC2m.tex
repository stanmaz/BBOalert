%%%%%%%%%%%%%%%%%%%%%%%%%%%%%%%%%%%%%%%%%%%%%%%%%%%%%%%%%%%%%%

\chapter{L'ouverture de 2\T}

\begin{multicols}{2}



L'ouverture de 2\T indique au moins 5 cartes à trèfle et entre 12 et 14H. Pas de majeure 4\ieme.
Attention, dans cette zone, les mains 5-3-3-2 s'ouvrent systématiquement de 1SA.
Les bicolores mineurs 5-4-2-2 de 12-13H s'ouvrent systématiquement de 1SA de même que les unicolores 6-3-2-2. De sorte que l'ouverture de 2\T garanti 14H ou un singleton.

Avec une majeure 5\ieme et une main au moins limite de manche, le répondant utilise un relai à 2\K.

Avec une solide majeure 6\ieme dans une main limite, le répondant propose de jouer 3 ou 4 dans sa majeure en la nommant au niveau de 3. Le contrat de 3SA est exclu. Un retour dans la mineure est exclu sauf cas d'espèce.

La logique du système est la suivante : dès qu'un fit mineur est connu au palier de 3, si l'ouvreur reparle, soit parce que la situation est forcing, soit parce qu'il est maximum dans une situation limite de manche, l'ouvreur nomme son singleton. On nomme toujours la couleur du singleton mais, il arrive que, faute de place, un des singletons soit nommé indirectement.

Cette ouverture est censé être un point fort du système. Il quelques petits détails à mémoriser mais leur oubli ne sera pas forcément dramatique.

\enchbox{Ouverture de 2\T}
{
2\K & 10H+ & Relai \\
2\C & 3-8H & 6 belles cartes à \C \\
2\P & 3-8H & 6 cartes cartes à \P\\
2SA & 11H & Proposition\\
3\T & 5-9H & Au moins 3 cartes \\
3\K & 12H+ & Naturel \\
3\C & 9-11H & Belle couleur 6\ieme\\
3\P & 9-11H & Belle couleur 6\ieme\\
3\NT && Conclusion \\
4\T && Chelem \\
}



\subsection*{Le relai à 2\K}

L'enchère de 2\K relai est une sorte de troisième couleur forcing. Toutes les mains limites de manches fittées à trèfle ou possédant une majeure 5\ieme passe par ce relai. Avec 14H, l'ouvreur sait que la manche sera atteinte d'une façon ou d'une autre.
\\
\enchbox{2\T -- 2\K --}
{
2\C &12-14H & 3 cartes à \C \\
2\P &12-14H & 3 cartes à \P (pas à \C)\\
2SA &12-13H & 6-4 médiocre\\
3\T &12-14H & 7 cartes à \T \\
3\K &12-14H & Bicolore 6-5\\
3\C &14H& 6-4 singleton \C\\
3\P &14H& 6-4 singleton \P\\
3SA &14H& 2-2-4-5 ou 2-2-3-6\\
}






\subsection*{La séquence 2\T--2\K--2\C}

 La redemande de l'ouvreur à 2\C promet 3 cartes à \C. Il peut aussi avoir 3 cartes à pique.

 Si le répondant était intéressé par les piques, il les annonce. la redemande à 2\P est forcing un tour.

 L'enchère de 3\P est une convention 2012, elle fixe l'atout \P dans l'optique d'un chelem.

 Il y a deux façons d'annoncer un fit limite, 2\NT et 3\T. Cela permet d'orienter les \NT, on l'espère, de la meilleure main.
\\
 \enchbox{2\T--2\K--2\C}
 {
  2\P  && 5 cartes à \P\\
  2\NT && Fit \T limite \\
 3\T &10-11H & Fit \T limite\\
 3\K & 12H+& Fit \T forcing manche\\
 3\C &10-11H& Proposition à \C\\
 3\P && Chelem à \C\\}

\subsection*{La séquence 2\T--2\K--2\C--2\P}

 S'il est fitté l'ouvreur répond 3\K. Ce qui laisse la place au répondant de proposer la manche en disant 3\P ou d'explorer le chelem à 3\C.
 Non fitté, l'ouvreur utilise l'enchère de 2\NT avec 4 cartes à \K (un chelem est encore possible dans cette couleur) ou revient à 3\T dans 6 cartes.

 \textit{Facultatif : Avec 14H, l'ouvreur utilise les enchères de 3\C et de 3\P comme des échos des enchères de 2\NT et de 3\T.}

 Comme d'habitude l'enchère de 3SA montre une main 2-3-3-6 et donc de 14H.


 \enchbox{2\T--2\K--2\C--2\P}
 {
 2\NT & 12-13H& 1\P-3\C-4\K-5\T \\
 3\T  & 12-13H& 1\P-3\C-3\K-6\T \\
 3\K  & 12-14H& 3\P-3\C-1\K-6\T \\
 \rb -> & 3\C & Chelem à \P \\
 -> & 3\P & Proposition \\
 3\C & 14H & 1\P-3\C-4\K-5\T \\
 3\P & 14H& 1\P-3\C-3\K-6\T \\
 3\NT &14H & 2\P-3\C-3\K-6\T \\
 }

 \subsection*{La séquence 2\T--2\K--2\C--3\T}

 En disant 3\T, le répondant montre qu'il n'était pas intéressé par les majeures mais veut juste explorer la manche. Il n'a pas dit 2\NT, l'ouvreur doit donc se méfier. Avec 14H sans singleton, il peut conclure. Sinon, avec 14H toujours, il annonce son singleton, 3\K avec un singleton \K et 3\K avec un singleton \P.

 \subsection*{La séquence 2\T--2\K--2\C--3\K}

 Dans cette situation, l'ouvreur annonce toujours son singleton. 3\C avec un singleton \K (faute de place), 3\P avec un singleton \P et 3\NT sans singleton.

\subsection*{La séquence 2\T--2\K--2\P}

 La redemande de l'ouvreur à 2\P promet annonce 3 cartes à \P et dénie 3 cartes à \C, le répondant dispose de trois enchères propositionnelle 2\NT, 3\T et 3\P.

 On ne dispose plus de l'enchère forcing à 2\NT donc 3\K sert de fourre-tout pour les chelems nébuleux.


 \enchbox{La séquence 2\T--2\K--2\P}
 {
 2\NT &11H& Proposition \\
 3\T & 10-11H& Fit \T limite  \\
 3\K & 12H+ & Fit \T forcing \\
 3\C && Chelem à \P \\
 3\P &11H& Proposition \\
 }

  \subsection*{La séquence 2\T--2\K--2\P--3\K}

  Dans cette situation, l'ouvreur annonce toujours son singleton. 3\C avec un singleton \C, 3\P avec un singleton \K (faute de place) et 3\NT sans singleton.

\subsection*{La séquence 2\T--2\K--2\NT}

 L'ouvreur annonce son bicolore mineur. Il est toujours temps de conclure à 4\C ou 4\P avec une couleur jouable seul et des ambitions de chelem qui tombent à l'eau.

 Le répondant peut maintenant produire les enchères non forcing de 3\T ou de 3\K.

 Il peut démarrer les enchères de chelem en fixant l'atout à 4\T ou 4\K.

 Les enchères de 3\C et de 3\P sont des essais sans conviction vers 3\NT. La couleur annoncée est bien gardée et l'ouvreur avise. On peut imaginer un chelem si miracle (chicane dans l'autre majeure par exemple).

 \subsection*{La séquence 2\T--2\K--3\T}

 L'ouvreur annonce 7 cartes à \T.

 On dispose d'un gadget facultatif à 3\K pour connaître le singleton.

 \subsection*{La séquence 2\T--2\K--3\T--3\K}

 3\C et 3\P annoncent un singleton.

 3SA annonce une main 2-2-2-7 (le contrat sera bien joué de la bonne main le plus souvent)

 4\T montre une huitième carte et donc un singleton \K

 \subsection*{La séquence 2\T--2\K--3\K}

 En cas de misfit grave, le répondant peut proposer le contrat de 4\T. Toute autre enchère est forcing de manche. S'il veut jouer un chelem à carreau, le répondant fixe l'atout immédiatement en disant 4\K. Sinon, on part du principe que l'atout est trèfle.

\end{multicols}


\chapter{L'ouverture de 2\K}

\begin{multicols}{2}
L'ouverture de 2 \K indique au moins 5 cartes à carreau et entre 12 et 14H. Pas de majeure 4\ieme.
Attention, dans cette zone, les mains 5-3-3-2 s'ouvrent systématiquement de 1SA.
Les bicolores mineurs 5-4-2-2 de 12-13H s'ouvrent systématiquement de 1SA de même que les unicolores 6-3-2-2. De sorte que l'ouverure de 2\K garanti 14H ou un singleton.

Le problème des carreaux se pose ici. On perd la possibilité de s'arrêter à 2\C qui devient relai. De même après la réponse à 2\P sur le relai à 2\C, on perd l'enchère naturelle de 2SA qui devient un relai pour retrouver les \C.

A part cela, la logique de l'ouverture de 2\K, ainsi que les petites astuces, est la même que celle de l'ouverture de 2\T.

\enchbox{Ouverture de 2\K}
{
2\C  &10H+ & Relai ambiguë\\
2\P & 3-8H & 6 cartes cartes à \P\\
2SA & 11H & Proposition\\
3\T & 12H+ & Naturel\\
3\K & 5-9H & Au moins 3 cartes \\
3\C & 9-11H & Naturel, belle couleur 6\ieme\\
3\P & 9-11H & Naturel, belle couleur 6\ieme\\
}

\enchbox{2\K--2\C--}
{2\P &12-14H & 3 cartes à \P \\
2SA &12-14H & 3 cartes à \C (pas à \P)\\
3\T &12-14H &  4-5 cartes\\
3\K &12-14H &  7 cartes\\
3\C &14H& 5-5 singleton \C\\
3\P &14H& 5-5 singleton \P\\
3SA &14H& 2-2-4-5 ou 2-2-3-6 \\
}


 \subsection*{La séquence 2\K--2\C--2\P}

 Quand l'ouvreur annonce 3 cartes à \P, il peut encore avoir à cartes à \C. Pour le savoir, le répondant doit faire un relai à 2\NT.

 S'il est fitté, l'ouvreur redemande 3\K, ce qui laisse la place au répondant de proposer la manche à 3\C ou d'engager un chelem à \C en disant 3\P.

 \enchbox{2\K--2 \C--2\P}
 {
 2\NT && 5 cartes à \C \\
\rw -> & 3\T & Singleton \C \\
 \rw-> & 3\K & 3 cartes à \C \\
\rw -> & 3\NT & 3-2-6-2 (14H) \\
\rw -> & 4\K & 7+ cartes à \K \\
\rb 3\T & & Fit \K forcing \\
 \rw 3\K & & Fit \K limite\\
 \rb 3\C && Proposition \\
 \rw 3\P && Chelem à \C \\
 }

  \subsection*{La séquence 2\K--2\C--2\P--2\NT}

 3\T : non fitté, 3-1-5-4 ou 3-1-6-3 Faute de place, l'ambiguïté ne peut pas être levée. Le répondant peut faire une enchère non forcing à 3\K. Il peut même passer. Au delà, toutes les enchères sont forcing de manche.

 3\K : 3 cartes à \C, l'enchère de 3\C devient propositionnelle et celle de 3\P est une enchère de chelem.

 \subsection*{ La séquence 2\K--2\C--2\P--3\T}

 3\K = 7 cartes, 3\NT improbable

 3\C = singleton \C

 3\P = singleton \T

 3SA = 3-2-6-3 et 14H

 \subsection*{ La séquence 2\K--2\C--2\NT}

 L'ouvreur possède 3 cartes à \C.

 L'enchère de 3\T indique un fit \K forcing de manche (on cherchait les \P)

 L'enchère de 3\K est non forcing.

 L'enchère de 3\C est une proposition de manche.

 L'enchère de 3\P explore le chelem, convention 2012.

 On a perdu la possibilité de s'arrêter à 2\NT. Le contrat de 3\K reste un bon compromis en cas de misfit.

 \subsection*{ La séquence 2\K--2\C--3\T}

 L'enchère est ambiguë, le plus souvent, il s'agit d'un bicolore 6-4 mais avec un 5-5 moche, on ne peut pas imposer la manche. Il y a toujours 10 cartes en mineure.

 On peut explorer le chelem en fixant l'atout au niveau de 4.

 On peut passer ou donner une préférence à \K.

 Les enchères de 3\C et de 3\P cherchent à récupérer un contrat à \NT.

\end{multicols}
