\chapter*{Exemples}
{
\rowcolors{0}{bananamania}{bananamania}
\setdefaults{bidlong=on}

Voici le résultat du test d'un tournoi de club.

\begin{enumerate}

 \item Ouverture de 1\NT fort

\hand!{A76}{AQJ9}{AT2}{QT5} \quad \hand!{T85}{853}{Q5}{KJ642} \quad
\begin{biddingpair}
 1C & 1D \\
 1N \\
\end{biddingpair}

L'ouverture de 1\T promet 4 cartes à \C. Le relai à 1\K montre de 5 à 11H sans 4 cartes à \P. Le redemande à 1\NT équivaut à une ouverture de 1\NT fort.

L'adoption du \NT faible n'est pas sans inconvénients. Ici, l’adversaire a eu deux occasions de nommer les piques.

\item Bicolore

\hand!{AKT52}{}{K85}{AJ765} \quad \hand!{J}{A432}{JT97}{9843} \quad
\begin{biddingpair}
 1S & 1N \\
 3C \\
\end{biddingpair}

L'ouverture de 1\P est commune à presque tous les systèmes existants. Avec son singleton, Est va à la pêche au bicolore en utilisant le relai forcing de 1\NT. L'enchère de 3\T indique un bicolore au moins 5-5 dans une ouverture maximum.

Au box-office, les paires jouant 3\T+1 se sont vues attribuées la note de 90,59\%. On a sans doute inhibé un réveil à \C en enchérissant ainsi.

\item Misfit

\hand!{AT3}{AKJT82}{2}{432} \quad \hand!{J}{5}{AKJ9653}{JT85} \quad
\begin{bidding}%
  2H & 2S & X & 3S \\
  p & p & p \\
\end{bidding}

En ouvrant de 2\C, Ouest a relativement bien vendu sa main. Malgré sa sixième carte à \C, il parait prudent d'attendre le deuxième Contre pour annoncer 4\C. Bonne pioche.

Au box-office, 4\C-2 vaut 40.59\%. Si la défense s'avise de donner une levée, la note monte à 68,82\%. Faire chuter 3 \P ne demande pas un gros effort et rapporte 84,12\%.

Sur ouverture de 1\C et une intervention à 2\P, beaucoup de joueurs en Est se sont sentis des ailes et ont préférés l'enchère forcing de 3\K au contre d'appel. Après cela, la catastrophe était inévitable. Sur une ouverture de 2\C, limitée à 14H, il est plus facile pour Est de juger correctement sa main.

\item  Sans-atout faible

\hand!{AJT82}{AQ6}{K53}{T6} \quad \hand!{Q73}{K}{AJ84}{AJ972} \quad
\begin{biddingpair}
 1N & 2C\\
 2S & 4S\\
\end{biddingpair}

L'ouverture de 1\NT promet un jeu régulier de 12 à 14H. Cette ouverture peut comporter une majeure cinquième mais pas une majeure quatrième.
Le répondant interroge avec un Stayman. Quand il apprend que l'ouvreur possède 5 cartes à \P, il peut conclure.

Cette séquence est meilleure que la séquence 1\P--2\T--2\P--3\P--4\P produite à d'autres table. Cela dit, sur cette donne particulière, le flanc peut dormir tranquille.

\item Sans-atout fort

\hand!{KQ9}{Q72}{AKJ95}{74} \quad \hand!{J72}{AT94}{Q62}{KT2} \quad
\begin{biddingpair}
 1H & 1S \\
 1N & 3N\\
\end{biddingpair}

Cette séquence est équivalente à la séquence 1\NT--2\T--2\K--3\NT du SEF. Seuls les adeptes du double Stayman s'en sorte mieux avec la séquence
1\NT--2\NT--3\T--3\P--3\NT qui dévoile un peu moins la main de l'ouvreur.

Les trèfles étaient 6-2 et l'entame trèfle faisait chuter le contrat de deux levées pour un maigre 37,65\%. Quelques heureux s'en sortirent avec une de chute.

\item Sans-atout faible

\hand!{AQ93}{K64}{AT95}{75} \quad \hand!{4}{QJ75}{KJ32}{A864}\quad
\begin{biddingpair}
 1D & 1H \\
 1S & 1N \\
\end{biddingpair}

Avec 4 cartes à pique, il est interdit d'ouvrir de 1\NT. Ou ouvre de 1\K, enchère dont la fonction, justement, est de montrer ces 4 cartes. Le relai à 1\C montre de 5 à 11H. Pas d'espoir de manche donc. La redemande à 1\NT montrerait une ouverture de 1\NT fort. Avec une ouverture de 1\NT faible, on les annonce en Texas. L'enchère de 1\P est Puppet pour 1\NT.

Cette séquence est supérieure à la séquence 1\K--1\C--1\P--1\NT qui a malheureusement dévoilée les 4 cartes à \C du déclarant et, mais ça ne joue que pour l'entame, les 4 carreaux de l'ouvreur.

La feuille de route est très folklorique. Les déclarants font entre 7 et 10 levées quand ils jouent à \NT.



 \item Sans-atout fort

 \hand!{KT86}{KQ8}{73}{AQJ9} \quad \hand!{J7532}{AJ95}{AJT}{7} \quad
 \begin{biddingpair}
  1D & 2H \\
  2N & 4C \\
  4S \\
 \end{biddingpair}

  Ouest ouvre de 1\K pour montrer ses 4 cartes à \P. Est le soutien à 2\C montrant une main au moins limite de manche.
  Ouest ne peut pas se contenter de rectifier à 2\P (non forcing) qui montrerait une main faible. Sur 2\NT, le splinter est une petite tentative de chelem que l'ouvreur refuse immédiatement avec deux petits honneurs à \T. De plus, il a déjà montré sa force.

  Au final, la main du déclarant a été moins dévoilée qu'après une ouverture de 1\NT fort suivi d'un Stayman.


 \item Jeu fort à \C

 \hand!{KQ}{AQJT6}{Q2}{AJT5} \quad \hand!{9732}{3}{A98754}{87}\quad
 \begin{biddingpair}
  1C & 1D \\
  1H & 1S \\
  2N & \\
 \end{biddingpair}

 A partie de 15H, on ne peut pas ouvrir de 2\C, on ouvre de 1\T. Avec ce bel As, la réponse de 1\P serait assez pessimiste. Ouest hésite à redemander 2\C, forcing de manche mais il n'a que 5 cartes et il lui manque un point. L'enchère de 1\C est forcing de toute façon. La redemande à 1\P tire la sonnette d'alarme. Ouest se contente donc de proposer la manche et Est passe.

 Selon son talent, le déclarant peut gagner (62\%) ou chuter (25\%).

 \item Jeu fort indéterminé

 \hand!{AJ}{4}{AQ954}{AK973}\quad\hand!{T93}{AKQ9852}{T7}{J}\quad
 \begin{biddingpair}
  1H & 2D \\
  2S & 3H\\
  4H & \\
 \end{biddingpair}

 L'ouverture de 1\C promet 15H sans majeure quatrième. La réponse de 2\K est un Texas \C. Sans enchère naturelle, l'ouvreur utilise l'enchère de 2\P comme Joker pour imposer une manche quelque part. Est propose un chelem à \C mais Ouest a déjà bien vendu sa soupe.

 \item Bicolore majeure

 \hand!{AQJ93}{AKT974}{}{84}\quad\hand!{742}{62}{KQ92}{AQJ6}\quad
 \begin{biddingpair}
  1C & 1S \\
  2H  & 2N \\
  3S & 4H \\
 \end{biddingpair}\quad
 \begin{biddingpair}
  1S & 2N \\
  4S&\\
 \end{biddingpair}

 Normalement, il faudrait 15H pour ouvrir de 1\T mais Ouest a jugé préférable d'ouvrir de 1\T car il maîtrise mieux la séquence.
 Beaucoup de sueur dans cette séquence.
 Cela dit s'il avait ouvert de 1\P, tout aurait été plus rapide !

 Sur ouverture de 1\P, Est fait un signal, soit de faiblesse, soit de platitude, en disant 1\P. Il a une main banale de manche. Ouest montre 5 cartes à \C.
 En reparlant, Est impose la manche. Conclure à 3\NT serait , hâtif, Ouest pourrait très bien avoir singleton \P. Cela permet à Ouest de montrer sa distribution exceptionnelle et de récupérer le fit in extremis.

 \item Contre de l'ouverture forte

 \hand!{A98}{QJ2}{KJ63}{KJ6}\quad\hand!{42}{43}{AQT97}{AT83}\quad
 \begin{bidding}
  1H & X & {\Redouble!} & 1S \\
  p & p & 2S & p \\
  2N & p & 3N & \\
 \end{bidding}

 \item Grand chelem

 \hand!{A2}{A}{KT}{AKQ97642}\quad\hand!{8654}{KT9}{A954}{T8}\quad
 \begin{biddingpair}
  1H & 1N \\
  3C & 3D\\
  4C & 4H\\
  7N & \\
 \end{biddingpair}

 Sur ouverture de 1\C, l'enchère de 1\NT indique 6-8H. Le saut à 3\T est forcing de manche. Est ne sait pas très bien où il en est et se contente de montrer qu'il ne craint pas le singleton carreau. Sur 4\T, la situation est plus claire. En matière de chelem mineur, Est-Ouest jouent le Turbo et 4\C montre le contrôle de la couleur et une carte clef. La conclusion est évidente et vaut 96\% au box-office.

 \item Une main banale pour finir (le reste des donnes est joué en Nord-Sud)

 \hand!{AJ95}{AKT64}{K4}{AQ}\quad\hand!{QT6}{J532}{Q8}{J974}\quad
 \begin{biddingpair}
  1C & 1S \\
  2H & 3H \\
  4H & \\
 \end{biddingpair}

 L'enchère de 1\P est une alerte jeu faible. C'est le seul cas où le soutien peut être différé en Trèfle Rouge. Ouest n'a plus les moyens d'imposer la manche et se contente de monter 5 cartes à \C. Est, finalement, est moins nul qu'on aurait pu le croire et fait un effort. Ouest récompense cet effort.



  \end{enumerate}}
